\section{MicrosoftPower BI}
Microsoft Power BI es una plataforma de visualización de datos utilizada principalmente con fines 
de inteligencia empresarial. Power BI son las siglas de Power Business Intelligence y se refiere a 
un conjunto de herramientas de software y conectores que le ayudan a transformar datos de múltiples 
fuentes en información procesable. \cite{power-bi-visualization}


\subsection{Carácteristicas}

\subsubsection{Usabilidad}
La interfaz amigable de Power BI y su fácil navegación permiten que los usuarios de negocios no técnicos 
tengan una curva de aprendizaje simple, lo que les permite crear visualizaciones de datos con unos pocos clics y sin experiencia en programación.

\subsubsection{Asequibilidad.} 
Power BI se ofrece a través de diferentes planes de precios para que las empresas de todos los tamaños 
puedan aprovechar sus capacidades sin preocuparse demasiado por las limitaciones presupuestarias.

\subsubsection{Visualizaciones.} 
Los líderes empresariales utilizan la narración de datos y las visualizaciones visualmente atractivas 
que les ayudan a comunicarse con clientes, empleados y otras partes interesadas.

\subsubsection{Personalización e interactividad.}
Power BI permite a los usuarios personalizar e interactuar con gráficos, tablas y otras 
visualizaciones que muestran sus datos. Estas características son útiles de diversas maneras para muchos usuarios diferentes. \cite{microsoft-power-bi}


\subsection{Componentes de Power BI}

\subsubsection{Power Query}
Power Query es el motor de transformación y combinación de datos. Te permite descubrir, conectar, 
combinar y refinar fuentes de datos para satisfacer tus necesidades de análisis. Se puede descargar 
como un complemento para Excel o se puede utilizar como parte de Power BI Desktop.

\subsubsection{Power Pivot}
Power Pivot es una técnica de modelado de datos que te permite crear modelos de datos, establecer relaciones 
y realizar cálculos. Utiliza el lenguaje de Expresiones de Análisis de Datos (DAX) para modelar datos simples y complejos.

\subsubsection{Power View}
Power View es una tecnología disponible en Excel, SharePoint, SQL Server y Power BI. Te permite crear gráficos, 
mapas y otras visualizaciones interactivas que dan vida a tus datos. Puede conectarse a fuentes de datos y filtrar 
datos para cada elemento de visualización o para todo el informe.

\subsubsection{Power Map}
Power Map de Microsoft para Excel y Power BI es una herramienta de visualización de datos en 3D que te permite mapear 
tus datos y trazar visualmente más de un millón de filas de datos en mapas de Bing en formato 3D desde una tabla de Excel 
o un Modelo de Datos en Excel. Power Map funciona con mapas de Bing para obtener la mejor visualización basada en información.

\subsubsection{Power BI Desktop}
Power BI Desktop es una herramienta de desarrollo para Power Query, Power Pivot y Power View. Con Power BI Desktop, 
tienes todo bajo la misma solución, lo que facilita el desarrollo de experiencias de BI y análisis de datos.

\subsubsection{Power Q\&A}
La función de preguntas y respuestas en Power BI te permite explorar tus datos en tus propias palabras. Es la forma más 
rápida de obtener una respuesta de tus datos utilizando lenguaje natural. Un ejemplo podría ser: ¿Cuáles fueron las ventas 
totales del año pasado? Una vez que hayas construido tu modelo de datos y lo hayas desplegado en el sitio web de Power BI, 
puedes hacer preguntas y obtener respuestas rápidamente. \cite{what_is_power_bi}
