\section{Benchmark}

En el cuadro 1 podemos observiar la comparativa completa de las 3 herramientas.

\begin{table*}[h!]
    \centering
    \caption{Comparación entre Power BI, AWS QuickSight y Looker Studio}
    \renewcommand{\arraystretch}{1.5} % Aumenta el espacio entre filas
    \setlength{\tabcolsep}{0.7\tabcolsep} % Aumenta el espacio entre columnas
    \begin{tabular}{|l|c|c|c|c|}
        \hline
        \textbf{Característica} & \textbf{Power BI} & \textbf{AWS QuickSight} & \textbf{Looker Studio} & \textbf{Ganador} \\ \hline
        \textbf{Integración con ecosistemas} & Sí & Sí & Sí & Todas \\ \hline
        \textbf{Interfaz de usuario} & Sí & Sí & Sí & Looker Studio \\ \hline
        \textbf{Manipulación de datos} & Sí & Sí & Sí & Power BI \\ \hline
        \textbf{Análisis de datos en tiempo real} & Sí & Sí & Sí & Looker Studio \\ \hline
        \textbf{Visualización de datos} & Sí & Sí & Sí & Looker Studio \\ \hline
        \textbf{Colaboración} & Sí & Sí & Sí & Power BI \\ \hline
        \textbf{Integración de datos} & Sí & Sí & Sí & Power BI \\ \hline
        \textbf{Modelado de datos} & Sí & Sí & No & Power BI \\ \hline
        \textbf{Precio} & No & Sí & Sí & Looker Studio / AWS QuickSight \\ \hline
        \textbf{Comunidad y soporte} & Sí & Sí & No & Power BI \\ \hline
    \end{tabular}
\end{table*}
    