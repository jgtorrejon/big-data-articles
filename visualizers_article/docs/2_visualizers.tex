\section{Data Visualization}
La visualización de datos es la representación de datos mediante el
uso de gráficos comunes, como diagramas, gráficos, infografías e 
incluso animaciones. Estas presentaciones visuales de información
comunican relaciones complejas entre datos y perspectivas basadas
en datos de una manera fácil de entender. \cite{ibm-data-visualization}


\subsection{Importancia}
Las empresas modernas suelen procesar grandes volúmenes de datos
procedentes de diversas fuentes, como las siguientes:

\begin{itemize}
  \item {\textbf{Sitios web internos y externos}}
  \item {\textbf{Dispositivos inteligentes}}
  \item {\textbf{Redes sociales}}
  \item {\textbf{Sistemas internos de recopilación de datos}}
\end{itemize}

Sin embargo, los datos sin procesar pueden ser difíciles de
comprender y utilizar. Por ello, los científicos de datos preparan
y presentan los datos en el contexto adecuado. Les dan una forma
visual para que los responsables de la toma de decisiones puedan
identificar las relaciones entre los datos y detectar patrones o
tendencias ocultas. La visualización de datos crea historias que
hacen avanzar la inteligencia empresarial y respaldan la toma de
decisiones basada en datos y la planificación estratégica.

\subsection{Beneficios}
Algunos beneficios de la visualización de datos son los siguientes:

\begin{itemize}
  \item {\textbf{Toma de decisiones estratégicas :}} Las partes interesadas
  clave y la alta dirección utilizan la visualización de datos para interpretarlos
  de forma significativa. Ahorran tiempo gracias a un análisis de datos más
  rápido y a la capacidad de visualizar el panorama general. Por ejemplo,
  pueden identificar patrones, descubrir tendencias y obtener información
  para mantenerse por delante de la competencia.
  \item {\textbf{Servicio al cliente mejorado :}} La visualización de datos
  resalta las necesidades y deseos de los clientes mediante una representación
  gráfica. Puede identificar deficiencias en su servicio al cliente, mejorar
  estratégicamente los productos o servicios y reducir las ineficiencias operativas.
  \item {\textbf{Mayor compromiso de los empleados :}} Las técnicas de visualización
  de datos son útiles para comunicar los resultados del análisis de datos a un
  equipo grande. Todo el grupo puede visualizar los datos en conjunto para
  desarrollar objetivos y planes comunes.
\end{itemize}

\subsection{Mejores prácticas}
Con tantas herramientas de visualización de datos disponibles, también ha aumentado
la visualización de información ineficaz. La comunicación visual debe ser simple
y deliberada para garantizar que la visualización de datos ayude a su público
objetivo a llegar a la información o conclusión deseada. Las siguientes prácticas
recomendadas pueden ayudar a garantizar que la visualización de datos sea útil
y clara:

\begin{itemize}
  \item {\textbf{Conozca a su audiencia :}} Piense para quién está diseñada su
  visualización y luego asegúrese de que la visualización de datos se ajuste a
  sus necesidades.
  \item {\textbf{Elija un elemento visual eficaz :}} Existen elementos visuales
  específicos diseñados para tipos específicos de conjuntos de datos.
  \item {\textbf{Manténgalo simple :}} Las herramientas de visualización de datos
  pueden facilitar la incorporación de todo tipo de información a su elemento visual.
  Sin embargo, el hecho de que pueda hacerlo no significa que deba hacerlo. En la
  visualización de datos, debe ser muy deliberado con respecto a la información
  adicional que agrega para centrar la atención del usuario.
\end{itemize}