\newpage
\section{Base de Datos}
\subsection{Recopilación}
Para este proyecto nosotros decidimos utilizar un dataset público
\textbf{IBM HR Analytics Employee Attrition and Performance} \cite{ibm-hr},
este data set se enfoca en la rotación de empleados en una empresa, analizando
varios aspectos de los empleados, como su salario, horas de trabajo, etc.

\subsubsection{Atributos}
El dataset cuenta con 35 atributos, los cuales son:
\begin{itemize}
    \item \textbf{Age:} Edad del empleado.
    \item \textbf{Attrition:} Si el empleado ha dejado la empresa o no.
    \item \textbf{BusinessTravel:} Frecuencia de viajes de negocios.
    \item \textbf{DailyRate:} Salario diario.
    \item \textbf{Department:} Departamento en el que trabaja el empleado.
    \item \textbf{DistanceFromHome:} Distancia de la casa al trabajo.
    \item \textbf{Education:} Nivel de educación.
    \item \textbf{EducationField:} Campo de estudio.
    \item \textbf{EmployeeCount:} Cantidad de empleados.
    \item \textbf{EmployeeNumber:} Número de empleado.
    \item \textbf{EnvironmentSatisfaction:} Satisfacción con el ambiente de trabajo.
    \item \textbf{Gender:} Género.
    \item \textbf{HourlyRate:} Salario por hora.
    \item \textbf{JobInvolvement:} Nivel de involucramiento en el trabajo.
    \item \textbf{JobLevel:} Nivel de trabajo.
    \item \textbf{JobRole:} Rol en la empresa.
    \item \textbf{JobSatisfaction:} Satisfacción con el trabajo.
    \item \textbf{MaritalStatus:} Estado civil.
    \item \textbf{MonthlyIncome:} Ingreso mensual.
    \item \textbf{MonthlyRate:} Ingreso mensual.
    \item \textbf{NumCompaniesWorked:} Número de empresas en las que ha trabajado.
    \item \textbf{Over18:} Si es mayor de 18 años.
    \item \textbf{OverTime:} Si trabaja horas extras.
    \item \textbf{PercentSalaryHike:} Porcentaje de aumento salarial.
    \item \textbf{PerformanceRating:} Calificación de desempeño.
    \item \textbf{RelationshipSatisfaction:} Satisfacción con las relaciones.
    \item \textbf{StandardHours:} Horas estándar.
    \item \textbf{StockOptionLevel:} Nivel de opciones de acciones.
    \item \textbf{TotalWorkingYears:} Años trabajados.
    \item \textbf{TrainingTimesLastYear:} Horas de entrenamiento el año pasado.
    \item \textbf{WorkLifeBalance:} Balance entre trabajo y vida personal.
    \item \textbf{YearsAtCompany:} Años en la empresa.
    \item \textbf{YearsInCurrentRole:} Años en el rol actual.
    \item \textbf{YearsSinceLastPromotion:} Años desde la última promoción.
    \item \textbf{YearsWithCurrManager:} Años con el actual jefe.
\end{itemize}


\subsection{Variables a analizar}
De todas las variables descritas anteriormente, nosotros decidimos analizar las
siguientes 9 variables:
\begin{itemize}
    \item \textbf{Age:} Edad del empleado.
    \item \textbf{MonthlyIncome:} Ingreso mensual.
    \item \textbf{NumCompaniesWorked:} Número de empresas en las que ha trabajado.
    \item \textbf{TotalWorkingYears:} Años trabajados.
    \item \textbf{YearsAtCompany:} Años en la empresa.
    \item \textbf{YearsInCurrentRole:} Años en el rol actual.
    \item \textbf{YearsSinceLastPromotion:} Años desde la última promoción.
    \item \textbf{YearsWithCurrManager:} Años con el actual jefe.
    \item \textbf{JobLevel:} Nivel de trabajo.
\end{itemize}