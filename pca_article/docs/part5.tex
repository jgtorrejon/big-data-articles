\newpage
\section{Preguntas adicionales}
\begin{enumerate}
    \item \textbf{¿Qué transformación de los datos es necesaria antes de aplicar PCA en Sklearn y por qué?}

    Previo a analizar los datos en Sklearn tenemos que realizar las siguientes tareas:
    \begin{itemize}
        \item Estandarización o normalización de datos.
        \item Tratamiento de valores faltantes.
        \item Verificación de linearidad.
    \end{itemize}

    Aplicar estas transformaciones es esencial para asegurar que el PCA capture correctamente la estructura de
    los datos y ofrezca resultados significativos.

    \item \textbf{¿Cómo evaluarías si el PCA fue exitoso para reducir la dimensionalidad en tu conjunto de datos?}

    Podemos determinar el éxito del PCA en la reducción de la dimensionalidad de los datos mediante la:
    \begin{itemize}
        \item Varianza explicada por cada componente principal.
        \item Rendimiento de un modelo supervisado.
        \item Reducción en el tiempo de procesamiento.
        \item Comparación de distancias o relaciones entre puntos.
    \end{itemize}

    \item \textbf{Si una variable presenta una correlación muy baja con todas las demás, ¿cómo afectará esto al PCA y qué acciones tomarías?}

    Su impacto en el PCA puede ser limitado, pero podría tener efectos en la interpretación de los componentes
    principales y en la eficiencia del modelo.

    Acciones que podríamos tomar:
    \begin{itemize}
        \item Eliminar la variable del análisis.
        \item Realizar un análisis de correlación más detallado.
        \item Realizar un análisis de componentes principales sin la variable.
    \end{itemize}

    \item \textbf{¿Qué problemas podrías encontrar si no limitas la cantidad de componentes principales a aquellos que expliquen una cantidad significativa de la varianza?}

    Al no limitar la cantidad de componentes principales podríamos enfrentar problemas como:
    \begin{itemize}
        \item Sobreajuste del modelo.
        \item Pérdida de interpretabilidad.
        \item Aumento de la complejidad computacional.
        \item Pérdida de generalización.
        \item Ruido en los datos transformados.
    \end{itemize}

    \item \textbf{¿Cómo podrías usar los resultados del PCA para mejorar un proceso dentro de tu empresa?}

    En este caso en particular para mejorar un proceso de recursos humanos con PCA podríamos:
    \begin{itemize}
        \item Identificación de factores clave de deserción.
        \item Segmentación de empleados en riesgo de deserción.
        \item Evaluación del impacto de las iniciativas de retención.
        \item Identificación de puntos críticos de mejora organizacional.
        \item Planificación de carrera y desarrollo profesional.
        \item Predicción de la rotación de empleados.
    \end{itemize}
\end{enumerate}