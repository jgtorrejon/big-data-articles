\newpage
\section{Análisis de Componentes Principales}
\subsection{Enunciado}
Aplicación del PCA: Utilizando Python y las librerías correspondientes (sklearn, pandas, matplotlib, etc.), aplicarán PCA sobre los datos estandarizados.
Análisis 1: Elección de Componentes Principales: Deberán determinar el número adecuado de componentes a retener basándose en la varianza explicada acumulada. Como pauta: retener los componentes principales que expliquen al menos el 80% de la varianza total. Esto suele dar un buen balance entre simplificación y retención de la información clave. En este punto se quedara con un número determinado de componentes según el criterio de la varianza total.
Análisis 2: Gráfica Unitaria y Bautizo de Ejes:
Una vez seleccionados los dos primeros componentes, generar una gráfica de dispersión (scatter plot) en dos dimensiones con los datos proyectados en el espacio definido por los primeros dos componentes principales. Interpretar qué representa cada eje en función de las variables originales. Basado en las cargas de las variables en cada componente, asignar nombres descriptivos a los ejes (por ejemplo, “Eficiencia Operacional” o “Costo de Producción”). En este punto se quedara con 2 componentes.
