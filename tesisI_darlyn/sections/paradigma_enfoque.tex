\chapter{Enfoque, Tipo y Diseño de Investigación}
El presente trabajo adopta un enfoque cuantitativo, dado que se orienta al diseño y construcción de una herramienta tecnológica concreta —un chatbot semántico multiperfil— que busca resolver un problema específico de acceso a documentación técnica y de negocio en entornos aeronáuticos. Este enfoque se fundamenta en una visión objetiva del conocimiento, centrada en la aplicación de técnicas de ingeniería de software, procesamiento de lenguaje natural y estructuración semántica de información, sin incluir interpretación subjetiva de los fenómenos sociales.

En cuanto al tipo de investigación, se clasifica como aplicada y descriptiva. Es aplicada porque busca resolver un problema real y específico dentro de una organización del sector aeronáutico, mediante el desarrollo de una solución tecnológica con impacto operativo. Es descriptiva porque tiene como propósito caracterizar la situación actual del acceso a la documentación en plataformas como Confluence, y detallar el funcionamiento, componentes y lógica del sistema desarrollado, sin establecer relaciones causales ni realizar pruebas estadísticas de eficacia.

Respecto al diseño metodológico, la investigación se encuadra dentro del diseño no experimental, específicamente de tipo transeccional descriptivo. Este diseño es coherente con el objetivo de documentar y explicar el desarrollo del sistema en un momento determinado del tiempo, sin manipulación de variables ni comparación entre grupos. La elección de este diseño responde al carácter tecnológico del trabajo, centrado en la producción de un artefacto funcional, más que en la medición de impactos o cambios sobre una población.

Este enfoque metodológico resulta pertinente dado el contexto organizacional donde se ejecuta la investigación. En entornos como el de Airnguru S.A., donde la documentación crítica se encuentra dispersa en plataformas digitales y es utilizada por diversos perfiles profesionales (desarrolladores, analistas, product managers), se requiere una solución técnica estructurada, capaz de modelar semánticamente grandes volúmenes de contenido. La prioridad está en el diseño, implementación y validación técnica de esta solución, y no en la evaluación estadística de su impacto en usuarios finales, al menos en esta etapa del proyecto.