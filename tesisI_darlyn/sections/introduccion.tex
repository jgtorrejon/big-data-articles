\chapter{Introducción}

En entornos organizacionales modernos, el acceso eficiente a la información es un factor crítico para la toma de decisiones, la productividad del equipo y la transferencia de conocimiento. Cuando los sistemas de información están fragmentados o carecen de estructura clara, los empleados invierten tiempo significativo en la búsqueda de datos relevantes, lo cual reduce la eficiencia organizacional y aumenta el riesgo de errores \parencite{davenport1997working}. Esta situación se agrava en contextos técnicos, donde la documentación dispersa puede ralentizar la incorporación de nuevos desarrolladores y obstaculizar el mantenimiento del software.
La implementación de mecanismos que permitan una recuperación de información basada en semántica y contexto, como los chatbots inteligentes, ha demostrado ser una solución eficaz para reducir las barreras de acceso al conocimiento. Según Hearst \parencite{hearst2009search}, los sistemas que incorporan búsqueda semántica y comprensión del lenguaje natural permiten a los usuarios formular consultas de manera más intuitiva y obtener respuestas relevantes, lo que resulta en una interacción más natural y productiva con la documentación \parencite{davenport1997working}.

Para los desarrolladores de software, el acceso rápido y claro a la documentación técnica es esencial para mantener la calidad del código, comprender sistemas existentes y acelerar el proceso de desarrollo. La falta de documentación accesible no solo incrementa el tiempo necesario para resolver problemas o incorporar nuevas funcionalidades, sino que también puede derivar en decisiones incorrectas sobre la arquitectura del sistema o el uso de bibliotecas \parencite{spinellis2006code}. En ambientes corporativos donde los sistemas evolucionan constantemente, los desarrolladores dependen de documentación actualizada para evitar errores, reducir la dependencia de expertos clave y facilitar el trabajo colaborativo en equipos distribuidos.
Una buena documentación actúa como una extensión del conocimiento organizacional y es crítica en procesos como la revisión de código, auditorías técnicas y la incorporación de nuevos integrantes. Como indica \parencite{forward2002relevance}, los desarrolladores dedican una parte considerable de su tiempo a buscar información, por lo que mejorar la accesibilidad de esta documentación tiene un impacto directo en la productividad y satisfacción laboral del personal técnico.

En sectores altamente regulados y especializados como el aeronáutico, los sistemas de información y software deben alinearse con normativas y estructuras externas complejas, tales como las definidas por ATPCO \parencite{atpco2022}, IATA o los sistemas de distribución global (GDS). Esta documentación contiene terminología técnica densa, como "fare classes", "record structures" (record 1, record 2, etc.) o "filing instructions", que son fundamentales para comprender y desarrollar correctamente soluciones de pricing y distribución aérea \parencite{atpco2022}.
Cuando los desarrolladores y product managers no tienen acceso claro y contextual a esta información, corren el riesgo de malinterpretar los requisitos del sistema o de omitir aspectos críticos del negocio. La integración de estas fuentes externas dentro de una interfaz conversacional inteligente facilita la comprensión progresiva del dominio, reduce errores y acelera la alineación entre tecnología y objetivos comerciales. Como destaca \parencite{sriraman2021semantic}, un sistema de acceso semántico puede actuar como puente entre los términos del negocio y su aplicación técnica en sistemas complejos.

Herramientas como Confluence se han convertido en estándares de facto para centralizar documentación técnica y de negocio dentro de muchas organizaciones. Su facilidad para integrar texto, código, imágenes y enlaces la hace una opción atractiva para documentar sistemas complejos. Sin embargo, su estructura basada en páginas y espacios tiende a fragmentar la información, dificultando la navegación y la recuperación eficiente del conocimiento almacenado, especialmente cuando crece en escala \parencite{morville2006ambient}.

A pesar de su flexibilidad, los usuarios enfrentan desafíos para encontrar información específica sin conocer previamente su ubicación exacta o los términos precisos utilizados para etiquetarla. Esta situación lleva a una sobrecarga cognitiva en usuarios técnicos y no técnicos, quienes deben invertir tiempo significativo en búsquedas manuales, lectura extensa y navegación no lineal \parencite{white2016content}. Además, la falta de uniformidad en la nomenclatura y estructura entre equipos complica aún más el acceso efectivo al conocimiento organizacional.

Una solución emergente a este problema es el uso de técnicas de Procesamiento de Lenguaje Natural (PLN), que permiten modelar semánticamente el contenido, identificar conceptos clave, y facilitar una interacción conversacional con los usuarios. La integración de PLN en entornos de documentación técnica ha demostrado mejorar la accesibilidad, reducir la fricción en la búsqueda de información y proporcionar respuestas contextualizadas a distintos perfiles de usuarios \parencite{jurafsky2023speech}.

La presente tesis se enmarca dentro del Área de Ciencias de la Computación Aplicadas, definida por la Facultad de Ingeniería en Ciencias de la Computación y Telecomunicaciones (FICCT) como el campo orientado al desarrollo de modelos, prototipos y nuevas formas de acceder al conocimiento computacional aplicado a contextos organizacionales y productivos. En particular, este trabajo se sitúa en la Línea de Sistemas Cognitivos, específicamente en el eje temático de Sistemas de Inteligencia Artificial, dado que propone el desarrollo de un chatbot basado en técnicas de Procesamiento de Lenguaje Natural (PLN) para la recuperación semántica de información técnica. Este enfoque contribuye a resolver una problemática real en entornos corporativos donde el acceso efectivo al conocimiento disperso es fundamental para la productividad y la toma de decisiones, en concordancia con los objetivos de la Unidad de Posgrado de la FICCT \parencite{ficct2022lineas}.


\section{Antecedentes del Problema}

En el entorno empresarial actual, el acceso ágil a la información técnica y de negocio se ha convertido en un factor esencial para el funcionamiento eficiente de los equipos multidisciplinarios. Estudios clásicos como el de Davenport y Prusak (1997) han evidenciado que cuando la información está fragmentada o desorganizada, los colaboradores dedican un tiempo considerable a su búsqueda, lo que impacta negativamente en la productividad. Esta situación se agudiza en entornos técnicos, donde la documentación dispersa dificulta la incorporación de nuevos desarrolladores y la comprensión de sistemas complejos.

En el ámbito del desarrollo de software, contar con documentación accesible y clara es fundamental para el mantenimiento de la calidad del código, la incorporación de nuevas funcionalidades y la toma de decisiones técnicas. Según Spinellis (2006), la carencia de documentación adecuada puede llevar a errores de diseño o implementación, y aumenta la dependencia de expertos clave. Forward y Lethbridge (2002) complementan esta visión al destacar que una porción significativa del tiempo de los desarrolladores se destina a la búsqueda de información relevante.

El desafío se intensifica en sectores especializados y regulados como el aeronáutico, donde los sistemas de software deben alinearse con estándares complejos definidos por organizaciones como ATPCO. La terminología técnica densa y especializada —por ejemplo, estructuras “record1”, “record2”, o clases tarifarias— representa una barrera significativa tanto para desarrolladores como para product managers cuando no existe un acceso claro, contextual y centralizado a dicha información (ATPCO, 2022). Esta falta de acceso puede derivar en errores funcionales o incomprensión de aspectos críticos del negocio.

Aunque herramientas como Confluence son ampliamente utilizadas para centralizar documentación, su modelo basado en páginas dispersas presenta limitaciones en cuanto a organización, búsqueda eficiente y comprensión semántica del contenido. Según Morville (2006) y White (2016), estas plataformas generan una sobrecarga cognitiva en los usuarios al dificultar la localización de información si no se conocen previamente los términos exactos o la estructura del repositorio.

Frente a este contexto, el uso de técnicas de Procesamiento de Lenguaje Natural (PLN) se ha posicionado como una solución emergente para mejorar el acceso al conocimiento organizacional. Jurafsky y Martin (2023) argumentan que el PLN permite modelar semánticamente el contenido técnico, facilitar la comprensión de conceptos y habilitar interfaces conversacionales más intuitivas. En esta línea, Sriraman et al. (2021) destacan el potencial de los sistemas de búsqueda semántica como puentes entre el lenguaje del negocio y su aplicación técnica en sistemas complejos.

