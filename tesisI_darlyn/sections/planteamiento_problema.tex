\chapter{Situación Problemática}
Con base en entrevistas exploratorias realizadas a desarrolladores de software en la empresa Airnguru S.A. (Ver Anexo ~\ref{fig:arbol_problemas}), se identificaron las siguientes problemáticas recurrentes:

\begin{itemize}
  \item \textbf{Dispersión de la documentación}: La información técnica y de negocio se encuentra distribuida en múltiples páginas dentro de Confluence, sin una estructura semántica coherente.
  
  \item \textbf{Dificultad de búsqueda}: Los usuarios deben conocer el nombre exacto del término o ubicación de la página para acceder al contenido necesario, lo que dificulta el uso eficiente de la plataforma.
  
  \item \textbf{Sobrecarga cognitiva}: La navegación no estructurada y el volumen de información generan fatiga en usuarios técnicos y no técnicos al buscar datos relevantes.
  
  \item \textbf{Dependencia de expertos}: El conocimiento implícito en ciertos procesos o términos requiere consultar a miembros senior del equipo, ralentizando el trabajo y aumentando el riesgo de errores.
  
  \item \textbf{Falta de contextualización}: Los conceptos técnicos específicos del sector aeronáutico (como \textit{record1}, \textit{fareclass}, etc.) no están explicados de forma accesible para perfiles no técnicos como los product managers.
  
  \item \textbf{Baja eficiencia en la incorporación de nuevos integrantes}: La curva de aprendizaje es alta debido a la falta de acceso directo y comprensible a la documentación clave.
  
  \item \textbf{Ausencia de recuperación semántica}: No existen herramientas que permitan buscar términos de forma contextual ni comprender relaciones entre conceptos técnicos y de negocio.
\end{itemize}

Esta situación revela una brecha significativa entre el estado actual (documentación dispersa, difícil de usar, poco accesible) y el estado deseado (acceso eficiente, contextual y conversacional a través de tecnologías inteligentes). Esta brecha constituye una situación problemática concreta en el entorno organizacional de Airnguru S.A.


\section{Formulación del Problema}
¿Cómo afecta la dispersión y estructuración ineficiente de la documentación técnica y de negocio en plataformas como Confluence al acceso ágil y contextualizado de la información en entornos aeronáuticos en la gestión 2025?

\section{Objeto de Estudio}
El acceso a la documentación técnica y de negocio en plataformas digitales del sector aeronáutico.

\section{Campo de Acción}
La recuperación de información en plataformas Confluence en la empresa Airnguru S.A. mediante técnicas de procesamiento de lenguaje natural, gestión 2025.

\section{Justificación de la Investigación}

\subsection{Justificación Teórica}
Desde el ámbito teórico, la presente investigación se justifica por la necesidad de fortalecer el cuerpo de conocimiento en torno al acceso semántico a la documentación técnica en entornos organizacionales complejos, como el sector aeronáutico. Aunque existen estudios sobre recuperación de información y sistemas conversacionales, pocos trabajos abordan de manera integrada la aplicación de técnicas de procesamiento de lenguaje natural (PLN) para facilitar la interacción entre humanos y repositorios documentales altamente especializados. Esta investigación propone un aporte al estudio de los sistemas inteligentes aplicados a la gestión del conocimiento, contribuyendo al desarrollo de soluciones que abordan la complejidad semántica inherente al lenguaje técnico de negocio.

\subsection{Justificación Práctica}
Desde la perspectiva práctica, la investigación responde a una necesidad concreta y observable dentro de la empresa Airnguru S.A., donde los desarrolladores y product managers enfrentan dificultades para acceder rápida y contextualmente a documentación crítica almacenada en Confluence. El desarrollo de un chatbot semántico multiperfil permitirá disminuir el tiempo de búsqueda de información, mejorar la comprensión de términos especializados y optimizar los flujos de trabajo. Además, esta solución tiene potencial de ser replicada en otras organizaciones con estructuras documentales similares, fortaleciendo la transferencia de conocimiento técnico y la eficiencia operativa.

\subsection{Justificación Social}
En el ámbito social y económico, la implementación de un sistema que optimiza el acceso al conocimiento técnico puede generar impactos positivos en la productividad del equipo, lo cual se traduce en una reducción de costos operativos asociados a tiempos improductivos y errores por mala interpretación de documentación. Asimismo, una mayor eficiencia en la incorporación de nuevos miembros al equipo puede fortalecer la estabilidad laboral y el clima organizacional, aspectos clave en el desarrollo humano dentro del entorno tecnológico.

\subsection{Justificación Metodológica}
Metodológicamente, esta investigación propone un abordaje innovador al problema de la recuperación documental mediante el diseño de un sistema conversacional alimentado por técnicas de PLN, estructura semántica y clasificación inteligente de contenidos. Se ofrece un aporte metodológico al integrar ingeniería de software, modelado del conocimiento y tecnología de agentes conversacionales para superar las limitaciones de los métodos tradicionales de acceso documental, que dependen de estructuras de búsqueda rígidas, etiquetas manuales y navegación jerárquica poco intuitiva.

\subsection{Justificación Personal}
A nivel personal, la motivación para desarrollar esta investigación surge de la experiencia directa enfrentando dificultades en el acceso a la documentación dentro de un entorno técnico complejo y altamente regulado. Esta problemática no solo afecta mi desempeño, sino también el de colegas de distintas áreas. El deseo de aportar una solución práctica que pueda mejorar los procesos internos y reducir las barreras al conocimiento ha impulsado la elección de este tema como objeto de estudio para la presente tesis.
