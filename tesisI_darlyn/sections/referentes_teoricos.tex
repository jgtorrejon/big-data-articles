\chapter{Referentes Teoricos}
\section{Referencia a Núcleos Teóricos a Desarrollar en la Investigación}
\subsection{Objeto de Estudio}
El objeto de estudio de esta investigación, referido al acceso a la documentación técnica y de negocio en plataformas digitales del sector aeronáutico, se sustenta en los siguientes núcleos teóricos:

\textbf{Gestión del conocimiento organizacional:} se refiere a los procesos mediante los cuales se crea, comparte y aplica conocimiento dentro de una organización, destacando el modelo de creación de conocimiento de Nonaka y Takeuchi (1995).
  
\textbf{Procesamiento de Lenguaje Natural (PLN):} campo de la inteligencia artificial orientado al análisis y generación del lenguaje humano por parte de las máquinas, abarcando tareas como la comprensión semántica y la generación de respuestas automatizadas (Jurafsky \& Martin, 2021).
  
\textbf{Recuperación de información semántica:} comprende las técnicas y modelos para mejorar la precisión de búsqueda en sistemas documentales mediante el análisis contextual del contenido, más allá de coincidencias textuales (Baeza-Yates \& Ribeiro-Neto, 2011).
  
\textbf{Sistemas de ayuda inteligente:} estructuras conversacionales que facilitan la interacción con grandes volúmenes de datos de forma accesible y contextualizada para los usuarios.


\subsection{Campo de Acción}
Respecto al campo de acción, centrado en la recuperación de información en Confluence en la empresa Airnguru S.A., se identifican los siguientes núcleos teóricos:

\textbf{Sistemas de gestión de contenido (CMS):} herramientas como Confluence que permiten almacenar, estructurar y consultar información empresarial, pero presentan desafíos semánticos cuando su uso no está estandarizado.

\textbf{Ontologías y modelos semánticos:} representación formal del conocimiento dentro de un dominio específico, como el aeronáutico, lo que permite estructurar conceptos y relaciones para su uso por sistemas inteligentes (Gruber, 1993).
  
\textbf{Interacción humano-computadora:} principios para diseñar interfaces conversacionales centradas en el usuario, promoviendo accesibilidad, usabilidad y eficiencia.


\subsection{Cómo del Objetivo General}
Para alcanzar el objetivo general de la investigación, se debe considerar el desarrollo de un modelo tecnológico funcional, sustentado en los siguientes núcleos:


\textbf{Diseño de chatbots y agentes conversacionales:} principios, arquitecturas y patrones conversacionales para la implementación de asistentes virtuales orientados a tareas específicas (McTear, 2017).
  
\textbf{Aprendizaje automático aplicado al PLN:} uso de técnicas supervisadas y no supervisadas para mejorar la capacidad de comprensión y respuesta del chatbot ante consultas reales (Manning et al., 2008).
  
\textbf{Evaluación de sistemas de recuperación:} métricas como precisión, cobertura, utilidad y tiempo de respuesta, fundamentales para valorar el rendimiento técnico de la solución.


\section{Índice Tentativo del Marco Teórico}

\begin{enumerate}
  \item \textbf{Fundamentos de la Gestión del Conocimiento Organizacional}
  \begin{enumerate}
    \item Concepto de conocimiento organizacional
    \item Tipos de conocimiento: tácito y explícito
    \item Modelos de gestión del conocimiento (Nonaka y Takeuchi)
    \item Relevancia del conocimiento en entornos tecnológicos complejos
  \end{enumerate}

  \item \textbf{Procesamiento de Lenguaje Natural (PLN)}
  \begin{enumerate}
    \item Definición y evolución del PLN
    \item Principales tareas del PLN en contextos organizacionales
    \item Aplicaciones del PLN en la recuperación de información
    \item Herramientas y bibliotecas comunes para el PLN
  \end{enumerate}

  \item \textbf{Sistemas de Recuperación de Información Semántica}
  \begin{enumerate}
    \item Definición y fundamentos teóricos
    \item Búsqueda basada en texto vs. búsqueda semántica
    \item Modelado semántico de documentos técnicos
    \item Casos de aplicación en entornos documentales empresariales
  \end{enumerate}

  \item \textbf{Ontologías y Representación del Conocimiento}
  \begin{enumerate}
    \item Definición de ontología en ciencias de la computación
    \item Estructura y componentes de una ontología
    \item Ontologías en dominios técnicos y aeronáuticos
    \item Rol de las ontologías en sistemas de recuperación semántica
  \end{enumerate}

  \item \textbf{Chatbots y Sistemas Conversacionales Inteligentes}
  \begin{enumerate}
    \item Evolución de los agentes conversacionales
    \item Arquitectura y componentes de un chatbot semántico
    \item Tipos de chatbots: basados en reglas, intencionales y de PLN
    \item Aplicación de chatbots en entornos corporativos
  \end{enumerate}

  \item \textbf{Interacción Humano-Computadora y Accesibilidad de la Información}
  \begin{enumerate}
    \item Principios de diseño centrado en el usuario
    \item Usabilidad en sistemas de recuperación documental
    \item Modelos de experiencia de usuario (UX) para interfaces conversacionales
    \item Accesibilidad por perfiles: técnicos y no técnicos
  \end{enumerate}
\end{enumerate}
