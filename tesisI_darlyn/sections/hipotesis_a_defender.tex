\chapter{Construcción Teórica}
\section{Hipótesis a Defender}

Si se implementa un modelo de chatbot semántico multiperfil basado en procesamiento de lenguaje natural para el acceso a la documentación técnica y de negocio en plataformas Confluence, entonces se facilitará el acceso contextualizado y eficiente a la información por parte de desarrolladores y product managers en la empresa Airnguru S.A., durante la gestión 2025.

\section{Identificación de Variables}

\subsection{Variable Independiente}
\textbf{Implementación de un chatbot semántico multiperfil basado en procesamiento de lenguaje natural: }Esta variable hace referencia a la propuesta tecnológica que será desarrollada dentro de la investigación. Se compone de elementos como el modelo conversacional, la estructura de recuperación semántica, la integración con la plataforma Confluence y la adecuación para múltiples perfiles de usuarios (desarrolladores y product managers).

\subsection{Variable Dependiente}
\textbf{Acceso contextualizado y eficiente a la documentación técnica y de negocio: }Esta variable representa el efecto esperado de la implementación del chatbot, medido en términos de la facilidad con la que los usuarios acceden, comprenden y utilizan la información almacenada en Confluence, reduciendo el tiempo de búsqueda y aumentando la comprensión del contenido técnico y de negocio.
