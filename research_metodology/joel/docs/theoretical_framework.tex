\newpage
\section{Marco Teórico}

\subsection{Breve esbozo de la evolución e importancia del desarrollo móvil}
El desarrollo móvil ha evolucionado drásticamente desde los años 90 hasta convertirse en
un motor clave de la economía digital actual. En sus inicios, los dispositivos móviles estaban
limitados a funciones básicas, y las pocas aplicaciones disponibles venían preinstaladas por
los fabricantes. Un ejemplo icónico es el juego Snake en los teléfonos Nokia, que marcó el inicio
del interés por los servicios interactivos en dispositivos portátiles.\\

Con el lanzamiento del iPhone en 2007 y la inauguración de la App Store en 2008, el panorama del
desarrollo móvil cambió radicalmente. Por primera vez, los desarrolladores podían distribuir
aplicaciones a una base global de usuarios. Este modelo fue rápidamente adoptado por Android con
su Play Store en el mismo año, lo que dio lugar a una explosión de aplicaciones móviles.
Para 2015, el número de aplicaciones disponibles superaba los 3 millones en ambas plataformas
principales. Gráficamente, se puede observar un crecimiento exponencial en el número de
aplicaciones disponibles desde 2008 hasta la actualidad, destacando el impacto del ecosistema
de tiendas de aplicaciones.\\

En términos de tecnología, el desarrollo inicial de aplicaciones móviles se centró en enfoques
nativos. Los desarrolladores utilizaban Objective-C o Swift para iOS y Java o Kotlin para Android,
logrando un rendimiento óptimo y acceso completo al hardware. Sin embargo, este enfoque implicaba
altos costos y largos tiempos de desarrollo, ya que era necesario escribir y mantener código
separado para cada plataforma.\\

El auge de las tecnologías multiplataforma marcó un punto de inflexión en el desarrollo móvil
a partir de 2015. Frameworks como React Native, lanzado por Facebook, y Flutter, introducido por
Google en 2018, ofrecieron soluciones que permitieron a los desarrolladores escribir un único
código base y desplegarlo en múltiples plataformas. Este avance no solo redujo significativamente
los costos, sino que también acortó los ciclos de desarrollo. Según un estudio de \parencite{statista-hibrid-development}, el 37\% de
los desarrolladores móviles preferían frameworks multiplataforma debido a su
eficiencia y escalabilidad.\\

La importancia del desarrollo móvil en la actualidad es innegable. Se estima que, para 2024,
habrá más de 7.5 mil millones de usuarios de teléfonos móviles en el mundo, lo que convierte
a las aplicaciones móviles en una herramienta esencial para llegar a audiencias masivas. Además,
sectores como el comercio electrónico, la banca y la educación dependen cada vez más de
aplicaciones móviles para interactuar con sus clientes. Un gráfico de la distribución de
usuarios de aplicaciones móviles por sectores muestra cómo el comercio electrónico lidera
con un 24\%, seguido de redes sociales con un 22\% y entretenimiento con un 18\%.\\

Finalmente, la integración de tecnologías emergentes como la inteligencia artificial, el
Internet de las Cosas (IoT) y la realidad aumentada en aplicaciones móviles ha ampliado
aún más su importancia estratégica. Estas innovaciones no solo mejoran la experiencia
del usuario, sino que también generan nuevas oportunidades de negocio, destacando el
papel fundamental del desarrollo móvil en el futuro de la economía global.


\subsection{Barreras de entrada en el mercado de desarrollo móvil}
El desarrollo móvil presenta desafíos significativos para los nuevos actores, especialmente
en términos de costos y complejidad técnica. Estos factores limitan la accesibilidad al mercado,
dificultando que pequeñas empresas o desarrolladores individuales compitan con grandes corporaciones.\\

Una de las principales barreras es la complejidad técnica inherente al desarrollo móvil. Crear
aplicaciones para plataformas como iOS y Android requiere conocimientos avanzados en lenguajes
de programación específicos, como Swift para iOS y Kotlin o Java para Android. Además, los
desarrolladores deben dominar las guías de diseño específicas de cada plataforma, como las Human
Interface Guidelines de Apple o el Material Design de Google. Esto no solo demanda tiempo y
capacitación constante, sino que también encarece el proceso de desarrollo. Por ejemplo, estudios
estiman que el tiempo promedio para desarrollar una aplicación móvil funcional oscila entre 4 y 6 meses,
dependiendo de la complejidad del proyecto.\\

El alto costo inicial del desarrollo es otra barrera crucial. Según encuestas realizadas en 2023,
el costo promedio de desarrollo de una aplicación móvil sencilla se sitúa entre \$40,000 y \$60,000.
Si el proyecto incluye características avanzadas, como integración con sistemas backend o funciones
de realidad aumentada, este costo puede superar los \$150,000. Además, los costos de prueba y
optimización para asegurar la compatibilidad en una amplia gama de dispositivos añaden un gasto significativo.\\

A esto se suma la necesidad de equipos especializados. Mientras que las grandes empresas cuentan con
recursos para contratar equipos multidisciplinarios que abarcan diseño, desarrollo, pruebas y mantenimiento,
las pequeñas empresas y los desarrolladores independientes suelen enfrentarse a la carga de cubrir múltiples
roles. Esto no solo incrementa los tiempos de desarrollo, sino que también eleva el riesgo de errores
o incompatibilidades.\\

En respuesta a estas barreras, han surgido frameworks multiplataforma como React Native y Flutter, que
permiten reducir tanto los costos como el tiempo de desarrollo. Estas herramientas ofrecen la posibilidad
de escribir un único código base que se puede implementar en múltiples plataformas, disminuyendo la carga
técnica y financiera asociada al desarrollo móvil. Sin embargo, la adopción de estas tecnologías requiere
una inversión inicial en capacitación y adaptación, lo que supone un nuevo desafío para quienes ingresan al mercado.\\

En resumen, el desarrollo móvil sigue siendo una oportunidad prometedora, pero las barreras técnicas y
económicas destacan la importancia de soluciones innovadoras para democratizar el acceso a este sector.
Optar por tecnologías eficientes y estrategias optimizadas es esencial para superar estas limitaciones
y competir en un mercado en constante evolución.

\subsection{Reducción de Costos en el Desarrollo Multiplataforma para desarrollo móvil}
El desarrollo móvil ha sido tradicionalmente un proceso costoso y que consume mucho tiempo, especialmente
cuando se requiere el desarrollo de aplicaciones para múltiples plataformas. Cada plataforma (iOS y Android)
tiene sus propias herramientas, lenguajes de programación y guías de diseño, lo que significa que para
alcanzar una audiencia completa, las empresas deben invertir en el desarrollo y mantenimiento de dos versiones
separadas de la misma aplicación. Este enfoque no solo es costoso, sino también ineficiente. Aquí es
donde entra en juego el desarrollo multiplataforma, una solución que ha ganado popularidad al permitir
a los desarrolladores escribir un solo código base que funcione en varias plataformas simultáneamente,
lo que lleva a una reducción significativa en los costos y los tiempos de desarrollo.\\

Una de las principales ventajas de los frameworks multiplataforma, como React Native, Flutter e Ionic,
es que permiten compartir la mayor parte del código entre diferentes plataformas. Esto significa que,
en lugar de desarrollar dos aplicaciones completas, los desarrolladores solo necesitan escribir y
mantener un único conjunto de código, lo que reduce considerablemente el esfuerzo y los costos asociados.
Por ejemplo, en un desarrollo nativo tradicional, se requiere que el equipo de desarrollo mantenga dos
bases de código diferentes, lo que implica no solo un mayor número de programadores, sino también
duplicación de esfuerzos en pruebas y actualizaciones. Con las herramientas multiplataforma, este
costo se reduce considerablemente, ya que el mismo código puede funcionar en iOS, Android e incluso
en la web, sin necesidad de hacer grandes modificaciones.\\

El ahorro en tiempos de desarrollo es igualmente significativo. Tradicionalmente, el desarrollo de
aplicaciones nativas para ambas plataformas podría tomar entre 6 y 12 meses, dependiendo de la
complejidad. Con el uso de frameworks multiplataforma, este tiempo se reduce entre un 30\% y un 50\%, ya
que los desarrolladores solo deben escribir un código base que se adapta a ambos sistemas operativos.
Este ahorro en tiempo no solo reduce los costos, sino que también permite que las aplicaciones lleguen
al mercado más rápido, lo que es crucial para aprovechar oportunidades de negocio en un mercado móvil
en constante cambio.\\

Además de los ahorros directos en desarrollo y tiempo, el uso de herramientas multiplataforma también
optimiza el proceso de mantenimiento. En lugar de tener que actualizar y corregir errores en dos bases
de código separadas, los desarrolladores pueden realizar cambios en un solo lugar, lo que acelera la
implementación de actualizaciones y mejoras. Esto reduce aún más los costos operativos a largo plazo.\\

Los costos de prueba también se ven reducidos, ya que una sola base de código implica un número menor
de pruebas comparado con las aplicaciones nativas, que requieren pruebas en dispositivos separados
para iOS y Android. Con frameworks multiplataforma, el proceso de pruebas se centraliza, lo que
disminuye los costos relacionados con la calidad y validación del producto final.\\

Finalmente, el reducido costo de personal es otra ventaja significativa. Las herramientas multiplataforma
suelen requerir menos especialistas, ya que los desarrolladores que dominan tecnologías como JavaScript,
Dart (en el caso de Flutter) o HTML/CSS pueden trabajar en ambas plataformas, en lugar de requerir equipos
separados especializados en iOS y Android. Esto no solo reduce el costo por contratación, sino que
también mejora la eficiencia del equipo.

\subsection{Aplicaciones, Casos de Estudio y Tendencias futuras en el desarrollo multiplataforma para aplicaciones móviles}
El desarrollo multiplataforma ha demostrado ser una herramienta poderosa en el ámbito de las aplicaciones móviles, permitiendo
a las empresas aprovechar al máximo sus recursos mientras llegan a una audiencia más amplia. Cada vez más empresas adoptan
estas tecnologías debido a su eficiencia en costos y tiempos de desarrollo. En este contexto, explorar algunos casos de estudio
y las tendencias futuras proporciona una visión más clara de cómo se está transformando este campo y hacia dónde se dirige.\\

Uno de los casos más destacados de éxito en el uso de frameworks multiplataforma es Instagram. La popular aplicación de redes
sociales, que originalmente fue desarrollada como una aplicación nativa para iOS, migró a React Native para su versión en
Android. Esta transición permitió a los desarrolladores de Instagram mantener una sola base de código para ambas plataformas,
lo que aceleró las actualizaciones y mejoró la consistencia de la experiencia de usuario. Otro ejemplo es Airbnb, que adoptó
React Native para mejorar la eficiencia del desarrollo en sus aplicaciones móviles. La adopción de esta tecnología permitió
reducir el tiempo de desarrollo en más de un 30\% y optimizar el proceso de mantenimiento a largo plazo.\\

Otro caso relevante es Alibaba, la gigante del comercio electrónico, que también utiliza un enfoque multiplataforma con Flutter,
el framework de Google. La decisión de Alibaba de adoptar Flutter ha permitido a la empresa crear aplicaciones que funcionan no
solo en dispositivos móviles, sino también en escritorios y dispositivos inteligentes, ampliando su alcance y asegurando una
experiencia de usuario coherente en múltiples plataformas.\\

El futuro del desarrollo multiplataforma se perfila con algunas tendencias clave que continuarán evolucionando y ampliando
el alcance de estas tecnologías. Flutter, por ejemplo, ha experimentado un crecimiento explosivo y se espera que se convierta
en una de las herramientas dominantes en el desarrollo móvil. Con Google mejorando continuamente su ecosistema, Flutter
podría dominar tanto el desarrollo para aplicaciones móviles como para aplicaciones de escritorio y web, eliminando
las barreras entre las plataformas y ofreciendo una verdadera solución “escribe una vez, corre en todas partes”.\\

Otra tendencia que se está consolidando es el aumento de la integración de inteligencia artificial y aprendizaje automático
en las aplicaciones móviles. Frameworks como React Native y Flutter ya están comenzando a integrar herramientas que
facilitan la implementación de funciones como el reconocimiento facial, chatbots y recomendaciones personalizadas.
A medida que estas tecnologías se vuelven más accesibles, los desarrolladores podrán crear aplicaciones más inteligentes
y personalizadas, mejorando la experiencia del usuario.
