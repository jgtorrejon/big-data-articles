\section{Introducción}

La programación híbrida en aplicaciones móviles es una estrategia ideal para desarrollar
soluciones multiplataforma con un presupuesto reducido. Este enfoque permite crear una
única base de código utilizando tecnologías web como HTML, CSS y JavaScript, que luego
se empaqueta en un contenedor nativo para ejecutarse en dispositivos Android e iOS.
Frameworks como Ionic, React Native o Flutter optimizan el proceso, permitiendo un
desarrollo ágil y reduciendo significativamente los costos y tiempos asociados con la
creación de apps separadas para cada plataforma.\\

Además de abaratar costos iniciales, las aplicaciones híbridas simplifican el mantenimiento
y las actualizaciones, ya que los cambios se realizan sobre un solo código y se reflejan en
todas las plataformas. Aunque pueden tener limitaciones en cuanto al rendimiento para
aplicaciones complejas, son ideales para startups o empresas que necesitan una app
funcional, atractiva y accesible en ambos sistemas operativos sin invertir en desarrolladores
especializados para cada ecosistema.\\

El tema está ubicado en el area 
Ciencias Generales de la Computación Aplicadas y la 
línea Ingeniería de software con un eje de 
Aspectos económicos y de negocio del proceso de desarrollo del software porque los costos
de desarrollo de software son un factor crítico en la toma de decisiones de las empresas.
