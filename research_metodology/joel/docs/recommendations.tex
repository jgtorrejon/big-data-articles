\newpage
\section{Recomendaciones}

\textbf{1. Selecciona el framework adecuado según las necesidades del proyecto}

Antes de optar por una solución multiplataforma, evalúa los requerimientos técnicos y funcionales del proyecto. Frameworks como Flutter son ideales para aplicaciones con interfaces complejas y animaciones fluidas, mientras que React Native ofrece mayor flexibilidad para integraciones nativas. Realiza pruebas piloto para validar el desempeño del framework elegido en escenarios reales.
	
\textbf{2. Prioriza proyectos con requisitos generales y menor dependencia nativa}

La programación multiplataforma es más efectiva en aplicaciones que no necesitan un acceso intensivo a funciones específicas del hardware, como gráficos avanzados, procesamiento intensivo o personalización a nivel del sistema operativo. Si el proyecto implica funcionalidades altamente especializadas, considera combinar desarrollo nativo con soluciones multiplataforma (aproximación híbrida).
	
\textbf{3.	Capacita al equipo en el uso de tecnologías multiplataforma}

Invierte en la formación del equipo de desarrollo para que dominen los frameworks multiplataforma y sus mejores prácticas. Esto incluye aprovechar comunidades activas, tutoriales oficiales y recursos de aprendizaje. Además, fomenta el uso de patrones de diseño reutilizables que faciliten la escalabilidad y el mantenimiento a largo plazo.
	
\textbf{4.	Planifica pruebas de compatibilidad y mantenimiento desde el inicio}

Aunque la multiplataforma permite compartir la mayor parte del código, los problemas de compatibilidad entre dispositivos y sistemas operativos pueden surgir. Implementa pruebas tempranas en diferentes entornos y dispositivos para garantizar una experiencia uniforme. Asimismo, aprovecha la centralización del código para agilizar las actualizaciones y reducir costos de mantenimiento.