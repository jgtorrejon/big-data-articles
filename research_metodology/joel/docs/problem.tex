\newpage
\section{Planteamiento del problema}

\subsection{Presentación de las problemáticas}
El desarrollo de aplicaciones móbiles es parte fundamental para las
empresas que buscan expandir su mercado y llegar a un público más amplio. Esto
se debe a que la mayoría de las personas cuentan con un dispositivo móvil y\
acceden a internet a través de él. Sin embargo, el desarrollo de aplicaciones
enfrenta problemáticas como:
\begin{itemize}
    \item \textbf{Mantenimiento de aplicaciones:} Las aplicaciones móviles
        requieren de mantenimiento constante para corregir errores y agregar
        nuevas funcionalidades.
    \item \textbf{Precios elevados:} El desarrollo de aplicaciones móviles
        puede ser costoso, especialmente para pequeñas y medianas empresas.
    \item \textbf{Dificultad para encontrar desarrolladores:} Encontrar
        desarrolladores con experiencia en el desarrollo de aplicaciones
        móviles puede ser complicado.
    \item \textbf{Dificultad para mantenerse actualizado:} El desarrollo de
        aplicaciones móviles requiere de estar al tanto de las últimas
        tecnologías y tendencias.
\end{itemize}

\subsection{Situación problemática}
El desarrollo de aplicaciones móviles es un proceso complejo que requiere de
conocimientos especializados y experiencia. Además, el desarrollo de
aplicaciones móviles puede ser costoso y requiere de mantenimiento constante
para corregir errores y agregar nuevas funcionalidades.\\

Estas limitantes agregan una barrera de entrada en el mercado de aplicaciones
móviles, especialmente para pequeñas y medianas empresas que no cuentan con
los recursos necesarios para desarrollar y mantener una aplicación móvil.

\subsection{Pregunta de investigación}
¿ Como puede el uso de tecnologías de desarrollo multiplataforma ayudar
a simplificar el desarrollo de aplicaciones móviles y abaratar costos?

