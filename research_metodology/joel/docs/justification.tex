\newpage
\section{Justificación}

\subsection{Justificación teórica}
El desarrollo de aplicaciones móviles enfrenta desafíos económicos
significativos debido a la necesidad de adaptar las soluciones a diferentes
plataformas, como Android e iOS. El enfoque tradicional de desarrollo nativo
implica la creación de aplicaciones separadas para cada sistema operativo,
utilizando lenguajes y herramientas específicas. Esto no solo incrementa los
costos relacionados con la contratación de desarrolladores especializados, sino
también los tiempos de desarrollo y mantenimiento.\\

En este contexto, el análisis de la programación multiplataforma emerge como
una alternativa teórica y práctica que busca mitigar estos desafíos
económicos.\\


\subsection{Justificación práctica}
Desde una perspectiva práctica, la programación multiplataforma ofrece
una solución directa a los problemas asociados con el desarrollo nativo
al reducir costos, tiempos y complejidad operativa. Empresas y
desarrolladores han adoptado frameworks multiplataforma debido a su
capacidad para generar aplicaciones funcionales y visualmente
atractivas a partir de un único código base.\\

Este enfoque elimina la necesidad de mantener equipos especializados
en múltiples lenguajes y herramientas, disminuyendo significativamente
el gasto en recursos humanos y tecnológicos.
