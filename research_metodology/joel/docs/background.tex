\section{Antecedentes}

Con el aumento significativo de usuarios de teléfonos móviles en la última
década, casi todas las empresas necesitan desarrollar una aplicación móvil.
Sin embargo, uno de los principales factores que debe tener en cuenta al
desarrollar una aplicación móvil es elegir entre el desarrollo de aplicaciones
móviles nativas o híbridas. \parencite{turing-hibrid-development} \\

Si bien ambas soluciones tienen pros y contras, muchas empresas optan por
las aplicaciones híbridas debido a sus beneficios en términos de presupuesto,
tiempo de comercialización y otros factores. Según un informe de Forbes \parencite{forbes-hibrid-development},
37 de las 50 principales aplicaciones minoristas en Estados Unidos son híbridas.
Además, plataformas populares como Twitter, Instagram, Gmail, Uber, etc.,
utilizan aplicaciones híbridas.\\

En respuesta a estas limitaciones, surgió la programación híbrida como una
alternativa más eficiente y accesible. Este enfoque permite desarrollar
aplicaciones utilizando tecnologías web estándares como HTML, CSS y JavaScript,
combinadas con frameworks que facilitan su integración con funcionalidades
nativas de los dispositivos. Frameworks como Ionic, React Native y Flutter
han demostrado su capacidad para reducir significativamente los costos
y los tiempos de desarrollo, manteniendo una experiencia de usuario satisfactoria.
Este proyecto se centra en analizar cómo la programación híbrida puede ser
una solución viable para optimizar recursos en el desarrollo de aplicaciones
móviles, especialmente en proyectos donde el presupuesto es un factor crítico.
