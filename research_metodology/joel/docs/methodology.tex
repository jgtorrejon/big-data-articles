\newpage
\section{Metodología}
\subsection{Enfoque de la investigación}
Este estudio adopta un enfoque cuantitativo para analizar cómo la programación
multiplataforma puede contribuir a la reducción de costos en el desarrollo de
aplicaciones móviles. A través de datos numéricos recopilados de empresas de
desarrollo de software, se compararán los costos asociados con tecnologías
multiplataforma, como Flutter y React Native, frente a las soluciones
nativas tradicionales. Se emplearán métricas como tiempos de desarrollo,
costos de mantenimiento, y cantidad de recursos necesarios para evaluar
objetivamente las diferencias entre ambos enfoques.\\

Para garantizar la validez de los resultados, se realizarán encuestas y estudios
de caso con desarrolladores y empresas del sector. Los datos obtenidos serán
procesados mediante análisis estadísticos para identificar patrones significativos
y establecer correlaciones entre la elección de herramientas multiplataforma
y la optimización de costos. Este enfoque permitirá generar conclusiones
basadas en evidencia objetiva, proporcionando una base sólida para la
toma de decisiones estratégicas en proyectos de desarrollo móvil.

\subsection{Diseño de la investigación}
El diseño metodológico de esta investigación es de tipo no experimental y
transeccional descriptivo, ya que no se manipularán las variables estudiadas,
sino que se observarán y analizarán tal como se presentan en su contexto
natural. El enfoque no experimental permite recolectar información sobre
el impacto de las tecnologías de programación multiplataforma en el
costo del desarrollo de aplicaciones móviles sin alterar las condiciones
existentes, asegurando así que los datos reflejen la realidad del sector.\\

El estudio se desarrollará de forma transeccional, lo que implica la
recopilación de datos en un único momento del tiempo. Este diseño permitirá
describir las características y tendencias actuales en el uso de
tecnologías multiplataforma como Flutter, React Native y Xamarin, así
como su influencia en la optimización de recursos y tiempos de desarrollo.
Los resultados se enfocarán en detallar patrones y correlaciones que
faciliten la comprensión del fenómeno y aporten información relevante
para las decisiones estratégicas en el desarrollo móvil.

\subsection{Población}
La población de este estudio está constituida por profesionales
especializados en el desarrollo de aplicaciones móviles dentro de startups
tecnológicas en Bolivia. En particular, se considera como población
objetivo a los tech leads y desarrolladores senior que participan
directamente en la implementación de tecnologías multiplataforma y
nativas en sus proyectos. Este grupo es clave para proporcionar
información técnica y estratégica relevante sobre los costos y
beneficios de ambas aproximaciones.\\

\subsection{Muestra}
La muestra estará conformada por 30 participantes seleccionados
de manera intencionada, distribuidos entre 3 tech leads y 3
desarrolladores senior de cada una de las 5 startups identificadas
para este estudio. Este tamaño muestral permite garantizar
la diversidad en las perspectivas y experiencias laborales,
al tiempo que facilita un análisis detallado de las prácticas
y decisiones relacionadas con el desarrollo móvil en un entorno
empresarial específico. Esta selección responde a la necesidad
de contar con información representativa del sector y generar
conclusiones aplicables al contexto boliviano.


\subsection{Técnicas e instrumentos}
Para la recolección de datos, se emplearán técnicas cuantitativas que
permitan obtener información objetiva y estandarizada sobre la percepción y
experiencia de los tech leads y desarrolladores senior en el uso de tecnologías
multiplataforma. Las principales técnicas utilizadas serán las encuestas
estructuradas y la revisión documental de proyectos desarrollados por
las startups seleccionadas.\\

El instrumento principal será un cuestionario diseñado específicamente para
este estudio, que incluirá preguntas cerradas y escalas tipo Likert.
Estas preguntas estarán orientadas a recolectar datos sobre costos de
desarrollo, tiempos de implementación, experiencia de usuario, y
facilidad de mantenimiento en proyectos multiplataforma y nativos.
Además, se utilizará una ficha de análisis documental para recopilar
información cuantitativa sobre los proyectos desarrollados por cada
startup, como el número de desarrolladores involucrados, el tiempo
total de desarrollo, y los costos asociados. Los instrumentos serán
validados a través de una prueba piloto aplicada a un grupo reducido
de participantes, asegurando su claridad y pertinencia para alcanzar
los objetivos del estudio.

\subsection{Elaboración del instrumento}
El instrumento principal, un cuestionario estructurado, fue diseñado tomando
en cuenta las necesidades del estudio y las características de la población
objetivo. Se desarrollaron preguntas cerradas y escalas tipo Likert para
garantizar la estandarización de las respuestas y facilitar su análisis
cuantitativo. Las dimensiones incluidas abarcan aspectos como costos,
tiempos de desarrollo, eficiencia en el mantenimiento y experiencia
del usuario en proyectos multiplataforma. Este cuestionario fue
sometido a una prueba piloto para verificar su validez y confiabilidad,
cuyos detalles se presentan en los anexos junto con el formato
final del instrumento.
