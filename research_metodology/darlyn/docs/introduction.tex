\begin{center}
    {\LARGE \textbf{INTRODUCCIÓN}} % Título centrado y en negrita
\end{center}

La estimación de costos en el desarrollo de software es un desafío inherente a la naturaleza intangible de este producto. En un mundo donde la economía digital impulsa la innovación y la competitividad, la incapacidad de calcular con precisión los costos de desarrollo puede llevar a pérdidas financieras, proyectos fallidos y relaciones comerciales insatisfactorias. A nivel global, empresas de todos los tamaños enfrentan la necesidad de establecer modelos más eficientes y estandarizados para valorar su trabajo, especialmente en un mercado donde la externalización y la globalización son cada vez más comunes. Este tema adquiere mayor relevancia 
con el auge de las metodologías ágiles y la inteligencia artificial, que prometen pero también complican la predicción de costos.

En Bolivia, donde el sector tecnológico está en crecimiento, el problema de la estimación de costos en el desarrollo de software adquiere un matiz especial. La falta de modelos estandarizados y herramientas locales para valorar los servicios de desarrollo de software lleva a una alta dependencia de las negociaciones subjetivas, lo que puede perjudicar tanto a los desarrolladores como a los clientes. Esto es especialmente importante en un contexto donde muchas empresas están comenzando a digitalizarse y dependen del software para competir en un mercado nacional e internacional. Abordar este desafío no solo contribuirá al fortalecimiento del sector tecnológico en Bolivia, sino que también permitirá que los desarrolladores y empresas logren acuerdos más justos y sostenibles, impulsando así el crecimiento de la economía digital en el país.

El tema está ubicado en el area 
Ciencias Generales de la Computación Aplicadas y la 
línea Ingeniería de software con un eje de 
Aspectos económicos y de negocio del proceso de desarrollo del software porque la estimación de costos afecta la 
rentabiliad de los proyectos del software, tanto para los desarrolladores
como para los clientes.