\section{Delimitación o Alcance Espacial}

El estudio se llevará a cabo en organizaciones y empresas dedicadas al desarrollo de software que aplican metodologías ágiles. La investigación se centrará en proyectos realizados en la ciudad de Santa Cruz de la Sierra, Bolivia, dado que esta región ha experimentado un crecimiento considerable en la industria tecnológica y en la adopción de metodologías ágiles para la planificación y ejecución de proyectos de software. 

El análisis se enfocará en equipos de desarrollo que utilizan herramientas específicas como \textit{puntos de historia} y \textit{Planning Poker}, permitiendo una evaluación contextualizada de su aplicación en el ámbito local. Aunque el alcance geográfico se limita a Santa Cruz, los resultados podrán servir como referencia para organizaciones con características similares en otras regiones.

\section{Alcance Temporal}

El alcance temporal del estudio abarca el análisis de proyectos de desarrollo de software realizados entre los años 2020 y 2024. Este período se selecciona debido a la creciente adopción de metodologías ágiles en la industria del software durante estos años, impulsada por la necesidad de adaptarse a entornos dinámicos y cambiantes. Además, este rango de tiempo permite evaluar cómo las técnicas de \textit{puntos de historia} y \textit{Planning Poker} han evolucionado y se han implementado en proyectos recientes, asegurando la relevancia y actualidad de los datos recopilados.
