\section{Objetivo General}

Analizar la efectividad de las técnicas de estimación ágiles, específicamente los \textit{puntos de historia} y el \textit{Planning Poker}, en la estimación de costos en proyectos de desarrollo de software, evaluando su impacto en la precisión, eficiencia y planificación de recursos dentro de equipos ágiles.

\section{Objetivos Específicos}

\begin{itemize}
    \item \textbf{Identificar} las principales características y fundamentos teóricos de las técnicas de estimación ágiles, como los \textit{puntos de historia} y el \textit{Planning Poker}.
    \item \textbf{Evaluar} la precisión de las estimaciones realizadas con \textit{puntos de historia} y \textit{Planning Poker}, comparándolas con métodos tradicionales de estimación.
    \item \textbf{Determinar} cómo la experiencia y composición del equipo afectan la efectividad de estas técnicas de estimación en proyectos ágiles.
    \item \textbf{Analizar} las principales limitaciones y desafíos asociados con el uso de estas herramientas en entornos de desarrollo de software.
    \item \textbf{Proponer} recomendaciones basadas en los hallazgos del análisis para optimizar la implementación de \textit{puntos de historia} y \textit{Planning Poker} en equipos ágiles.
\end{itemize}
