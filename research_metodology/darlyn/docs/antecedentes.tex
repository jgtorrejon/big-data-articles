\nochapterlabel
\chapter{Antecedentes}
La estimación precisa de costos en el desarrollo de software es un desafío constante en la ingeniería de software, especialmente en entornos ágiles. Técnicas como los \textit{Puntos de Historia} y el \textit{Planning Poker} han sido desarrolladas para abordar este problema, permitiendo a los equipos estimar el esfuerzo necesario para completar historias de usuario de manera colaborativa 
y basada en consenso \parencite{asana_planning_poker}
Los \textit{Puntos de Historia} son una unidad de medida abstracta que refleja el esfuerzo relativo, la complejidad y el riesgo asociados a una historia de usuario. Esta técnica permite a los equipos evaluar tareas en función de su tamaño relativo, en lugar de estimar en unidades de tiempo, lo que facilita 
la planificación y priorización en entornos ágiles \parencite{asana_story_points}.

Por otro lado, el \textit{Planning Poker} es una técnica de estimación que combina la opinión de todos los miembros del equipo para alcanzar una estimación consensuada. Durante una sesión de \textit{Planning Poker}, cada miembro selecciona de manera independiente una carta con el valor que considera adecuado para una tarea específica. Posteriormente, las discrepancias se discuten hasta alcanzar un consenso, promoviendo 
la colaboración y la precisión en las estimaciones \parencite{asana_planning_poker}.

A pesar de su efectividad en proyectos ágiles, estas técnicas enfrentan desafíos, como la subjetividad inherente al proceso y la falta de estandarización en algunos equipos. Sin embargo, su uso ha demostrado ser fundamental para mejorar la precisión de las estimaciones y la planificación 
en proyectos de software de diversa escala \parencite{asana_planning_poker}.
