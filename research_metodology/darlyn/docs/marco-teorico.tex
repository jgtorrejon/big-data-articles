\chapter{Marco Teorico}
\section{Antecedentes Históricos del Desarrollo de Software Ágil}
El desarrollo de software ágil tiene sus raíces en la década de 1990, como respuesta a los problemas asociados con los métodos tradicionales de desarrollo, comúnmente denominados como métodos en cascada. Estos métodos fueron criticados por su rigidez y falta de adaptabilidad a los cambios en los requerimientos del cliente. Según \textcite{beck2001manifesto}, el desarrollo ágil surge como un enfoque colaborativo, iterativo y adaptable.

Uno de los hitos clave en esta evolución fue la publicación del \textit{Manifesto for Agile Software Development} en 2001, que estableció los principios fundamentales de las metodologías ágiles. Estos principios incluyen la prioridad hacia la interacción humana, la colaboración con el cliente y la capacidad de responder rápidamente a los cambios \parencite{fowler2001agile}.

Otro avance importante fue la introducción de metodologías como \textit{Scrum} y \textit{Extreme Programming (XP)}, que promovieron la gestión iterativa de proyectos y las prácticas de desarrollo enfocado en la calidad del software \parencite{schwaber1997scrum}. Estas metodologías permitieron abordar los desafíos asociados con la incertidumbre y la complejidad en el desarrollo de software.

En conclusión, el desarrollo ágil se ha consolidado como un enfoque clave en la ingeniería de software moderna, destacándose por su capacidad para adaptarse a entornos cambiantes y centrarse en la satisfacción del cliente.

\section{Evolución de los Métodos de Estimación de Costos}
La estimación de costos en el desarrollo de software ha evolucionado significativamente desde los enfoques tradicionales hasta los métodos ágiles actuales. En los años 70 y 80, se popularizaron modelos como COCOMO (\textit{Constructive Cost Model}), que ofrecían estimaciones basadas en líneas de código y métricas cuantitativas \parencite{boehm1981cocomo}. Sin embargo, estos modelos eran rígidos y no contemplaban adecuadamente la incertidumbre en los requerimientos.

Con el tiempo, surgieron métodos basados en historias de usuario y esfuerzos relativos, como los \textit{puntos de historia}, utilizados ampliamente en entornos ágiles. Estos enfoques destacan por priorizar la colaboración y la adaptabilidad frente a los cambios \parencite{cohn2004userstories}. El uso de herramientas como \textit{Planning Poker} fomentó una mayor participación del equipo, haciendo las estimaciones más democráticas y contextuales \parencite{cohn2005agileestimating}.

Hoy en día, los métodos de estimación han integrado técnicas ágiles con herramientas de análisis predictivo, permitiendo a los equipos combinar datos históricos y metodologías colaborativas para generar estimaciones más precisas \parencite{larman2004agile}.

\section{Estado Actual del Uso de Puntos de Historia y Planning Poker}
En la actualidad, los \textit{puntos de historia} y el \textit{Planning Poker} se han consolidado como técnicas clave para la estimación de costos en proyectos de desarrollo de software bajo metodologías ágiles. Los \textit{puntos de historia} permiten a los equipos medir el esfuerzo relativo de tareas basándose en experiencias previas, promoviendo un enfoque iterativo y adaptable \parencite{cohn2005agileestimating}. Por otro lado, el \textit{Planning Poker} fomenta la colaboración y el consenso en las estimaciones, involucrando a todos los miembros del equipo en un proceso participativo \parencite{schwaber2020scrumguide}.

Estas técnicas son ampliamente utilizadas en empresas tecnológicas debido a su flexibilidad y facilidad de implementación. Sin embargo, su efectividad puede variar según el nivel de experiencia del equipo y la calidad de las historias de usuario \parencite{larman2004agile}. En entornos con alta incertidumbre o cambios frecuentes, estas herramientas han demostrado ser más precisas que los enfoques tradicionales de estimación \parencite{cohn2004userstories}.

A pesar de su éxito, el uso de \textit{puntos de historia} y \textit{Planning Poker} enfrenta críticas, como la subjetividad en las estimaciones y la falta de estandarización. Investigaciones recientes han explorado la integración de estas técnicas con herramientas de análisis predictivo para abordar estas limitaciones \parencite{boehm2020software}.

\section{Revisión de Herramientas y Técnicas de Planificación Ágil}
En el ámbito de las metodologías ágiles, las herramientas y técnicas de planificación juegan un papel fundamental para garantizar la adaptabilidad y eficiencia en la gestión de proyectos. Entre las técnicas más destacadas se encuentran los \textit{puntos de historia}, utilizados para medir el esfuerzo relativo de las tareas, y el \textit{Planning Poker}, que fomenta la colaboración del equipo en las estimaciones \parencite{cohn2005agileestimating}.

Además de estas técnicas, herramientas digitales como \textit{Jira}, \textit{Trello} y \textit{Asana} han facilitado la implementación de la planificación ágil, permitiendo a los equipos rastrear historias de usuario, gestionar tareas y priorizar el trabajo de forma visual \parencite{rubin2012essential}. Estas plataformas integran métricas ágiles, como la velocidad del equipo, que se calcula a partir de los puntos de historia completados en cada iteración \parencite{schwaber2020scrumguide}.

Otra técnica clave es la planificación basada en tableros Kanban, que complementa a Scrum y otros marcos ágiles, proporcionando visibilidad sobre el flujo de trabajo y facilitando la identificación de cuellos de botella \parencite{anderson2010kanban}. Estas herramientas y técnicas han transformado la forma en que los equipos gestionan proyectos, promoviendo una mayor transparencia y adaptabilidad.

A pesar de su efectividad, la elección de herramientas y técnicas depende del contexto del proyecto, el tamaño del equipo y la experiencia previa. Por lo tanto, una comprensión integral de estas herramientas es esencial para maximizar los beneficios de la planificación ágil.

\section{Ventajas y Desafíos de los Métodos de Estimación Ágil}
Los métodos de estimación ágil, como los \textit{puntos de historia} y el \textit{Planning Poker}, han transformado la forma en que los equipos abordan la planificación de proyectos en entornos dinámicos. Entre sus principales ventajas se encuentra la capacidad de adaptarse rápidamente a los cambios en los requerimientos del cliente, promoviendo una estimación iterativa y colaborativa \parencite{cohn2005agileestimating}. Esto reduce significativamente los riesgos asociados con los enfoques tradicionales de estimación, que suelen ser más rígidos y lineales \parencite{rubin2012essential}.

Otra ventaja destacada es la inclusión de todos los miembros del equipo en el proceso de estimación, lo que mejora la precisión al aprovechar diferentes perspectivas y niveles de experiencia \parencite{schwaber2020scrumguide}. Además, la simplicidad inherente de estas técnicas facilita su adopción en equipos con poca experiencia previa en metodologías ágiles \parencite{anderson2010kanban}.

Sin embargo, estos métodos no están exentos de desafíos. Uno de los problemas más comunes es la subjetividad en las estimaciones, ya que los puntos de historia y el \textit{Planning Poker} dependen en gran medida de la experiencia del equipo y de la claridad de las historias de usuario \parencite{larman2004agile}. Además, la falta de estandarización puede dificultar la comparación de estimaciones entre equipos o proyectos \parencite{boehm2020software}.

Otro desafío importante es la dificultad para incorporar datos históricos en el proceso de estimación ágil, lo que puede limitar su efectividad en proyectos con alta incertidumbre o equipos nuevos. A pesar de estas limitaciones, los métodos de estimación ágil continúan siendo una herramienta valiosa en la gestión de proyectos, especialmente cuando se complementan con herramientas de análisis predictivo o métricas automatizadas \parencite{rubin2012essential}.

\section{Normativas Internacionales en Desarrollo de Software}
El desarrollo de software está guiado por diversas normativas internacionales que establecen estándares de calidad, seguridad y gestión de proyectos. Una de las más reconocidas es la familia de normas ISO/IEC 12207, que define un marco para los procesos del ciclo de vida del software, desde su concepción hasta su retirada \parencite{iso12207}.

Otra normativa destacada es la ISO/IEC 25010, que introduce un modelo de calidad para sistemas y software, especificando características como usabilidad, confiabilidad y rendimiento \parencite{iso25010}. Estas normativas son fundamentales para garantizar la calidad del producto final y asegurar que los procesos de desarrollo cumplan con los requisitos del cliente y las regulaciones aplicables.

En términos de seguridad, la norma ISO/IEC 27001 proporciona directrices para implementar sistemas de gestión de la seguridad de la información (SGSI), un aspecto crucial en proyectos de software que manejan datos sensibles \parencite{iso27001}. Adicionalmente, la guía OWASP (Open Web Application Security Project) complementa estos estándares al enfocarse en la identificación y mitigación de vulnerabilidades en aplicaciones web \parencite{owasp2021}.

Estas normativas internacionales no solo ofrecen un marco común para los equipos de desarrollo, sino que también promueven la interoperabilidad entre sistemas, un factor clave en proyectos globales. Sin embargo, su adopción requiere una planificación cuidadosa para adaptarlas a las necesidades específicas de cada organización.

\section{Buenas Prácticas en Estimación de Proyectos según PMI}
El Project Management Institute (PMI) establece un conjunto de buenas prácticas para la gestión de proyectos, incluyendo directrices específicas para la estimación de costos y tiempos. Estas prácticas están recogidas en el \textit{Project Management Body of Knowledge} (PMBOK Guide), que es ampliamente reconocido como un estándar global en la gestión de proyectos \parencite{pmi2021pmbok}.

Entre las buenas prácticas recomendadas por el PMI se encuentra la estimación análoga, que utiliza datos históricos de proyectos similares para realizar proyecciones. Esta técnica es útil para obtener estimaciones rápidas en fases iniciales del proyecto \parencite{pmi2021pmbok}. Otra práctica clave es la estimación paramétrica, que emplea relaciones matemáticas entre variables del proyecto, como la cantidad de tareas y la productividad del equipo \parencite{kerzner2017projectmanagement}.

El PMI también enfatiza la importancia de la descomposición del trabajo en elementos más pequeños y manejables, conocida como \textit{Work Breakdown Structure} (WBS). Este enfoque facilita la asignación precisa de recursos y tiempos para cada actividad \parencite{pmi2021pmbok}.

Finalmente, las buenas prácticas del PMI promueven la colaboración entre las partes interesadas para validar las estimaciones y ajustarlas según sea necesario. Esto asegura que las proyecciones reflejen tanto las restricciones del proyecto como las expectativas de los interesados \parencite{lock2020projectmanagement}.

\section{Limitaciones Éticas y Técnicas en la Aplicación de Métodos Ágiles}
La implementación de métodos ágiles, aunque efectiva en muchos contextos, enfrenta limitaciones tanto éticas como técnicas que deben ser consideradas durante su aplicación. Desde una perspectiva ética, uno de los desafíos más significativos es el equilibrio entre la transparencia y la privacidad. En muchos equipos ágiles, la constante retroalimentación y las reuniones frecuentes, como las \textit{daily stand-ups}, pueden generar un ambiente de presión excesiva para los integrantes del equipo \parencite{cockburn2001agile}.

Otro aspecto ético relevante es la posible desigualdad en la participación del equipo. Los métodos ágiles, como el \textit{Planning Poker}, suponen un equilibrio entre las opiniones de los integrantes, pero en la práctica, miembros más experimentados o con mayor autoridad pueden influir desproporcionadamente en las decisiones \parencite{rubin2012essential}.

Desde una perspectiva técnica, las principales limitaciones incluyen la dependencia de la experiencia del equipo y la calidad de las historias de usuario. Las estimaciones basadas en \textit{puntos de historia} pueden ser subjetivas y variar significativamente entre equipos, dificultando la comparación o reutilización de métricas \parencite{larman2004agile}. Además, en proyectos complejos o con altos niveles de incertidumbre, los métodos ágiles pueden carecer de la estructura necesaria para manejar todos los aspectos técnicos del desarrollo \parencite{boehm2020software}.

A pesar de estas limitaciones, los métodos ágiles continúan siendo herramientas valiosas, siempre que se implementen de manera consciente, respetando principios éticos y adoptando técnicas que mitiguen sus desventajas técnicas.
