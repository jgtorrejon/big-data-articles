\chapter{Planteamiento del Problema}
\section{Presentación de las Problemáticas}

El enfoque ágil, específicamente mediante técnicas como los \textit{puntos de historia} y el \textit{Planning Poker}, busca mejorar la precisión en las estimaciones promoviendo la colaboración entre equipos y utilizando métricas iterativas y relativas. Sin embargo, estas metodologías enfrentan problemáticas clave, como:

\begin{itemize}
    \item \textbf{Subjetividad en las estimaciones:} Las valoraciones basadas en puntos de historia pueden variar significativamente según la experiencia y perspectiva de los miembros del equipo.
    \item \textbf{Dependencia de la experiencia del equipo:} La efectividad de \textit{Planning Poker} depende de un equipo con experiencia previa en proyectos similares, lo que limita su aplicabilidad en equipos menos maduros.
    \item \textbf{Dificultad en la estandarización:} Los puntos de historia y \textit{Planning Poker} no siempre son fácilmente comparables entre equipos o proyectos, lo que complica la evaluación del desempeño y la predicción futura.
    \item \textbf{Falta de datos históricos:} En proyectos nuevos o innovadores, la ausencia de datos históricos dificulta establecer referencias confiables para las estimaciones.
\end{itemize}


\section{Situación Problemática}

La estimación de costos y tiempos en el desarrollo de software es un componente crítico para garantizar el éxito de un proyecto. En entornos de trabajo tradicionales, estas estimaciones suelen realizarse utilizando métodos rígidos y lineales que no se adaptan fácilmente a los cambios frecuentes en los requerimientos. Sin embargo, con la creciente adopción de metodologías ágiles, han surgido técnicas como los \textit{puntos de historia} y el \textit{Planning Poker}, que buscan proporcionar mayor flexibilidad y precisión en estas estimaciones.

A pesar de su potencial, estas técnicas presentan múltiples desafíos en su implementación. Por ejemplo, las estimaciones realizadas con puntos de historia suelen ser subjetivas, dependiendo de la experiencia y el criterio de los miembros del equipo. Además, el uso de \textit{Planning Poker} requiere equipos bien entrenados y con experiencia, lo que puede dificultar su aplicación en proyectos con equipos nuevos o poco maduros.

Otro problema significativo es la falta de estandarización en estas técnicas. Los puntos de historia no son fácilmente comparables entre diferentes proyectos o equipos, lo que complica la evaluación del desempeño y la generación de predicciones confiables. Asimismo, en proyectos innovadores o en fases iniciales, la ausencia de datos históricos dificulta establecer referencias para realizar estimaciones precisas.

Estas problemáticas subrayan la importancia de analizar y evaluar la efectividad de los \textit{puntos de historia} y el \textit{Planning Poker} como enfoques para la estimación de costos y tiempos en entornos ágiles. Comprender sus limitaciones y áreas de mejora puede contribuir al desarrollo de estrategias más efectivas para la planificación y ejecución de proyectos de software.

\section{Pregunta de Investigación}
¿Cómo impacta el uso de puntos de historia y Planning Poker en la precisión y eficiencia de la estimación de costos en proyectos de desarrollo de software bajo metodologías ágiles?