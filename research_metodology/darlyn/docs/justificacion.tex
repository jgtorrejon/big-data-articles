\section{Justificación Teórica}

La estimación de costos y tiempos en el desarrollo de software es una actividad esencial dentro de la ingeniería de software, dado su impacto en la planificación y ejecución de proyectos. Desde una perspectiva teórica, las metodologías ágiles han emergido como una respuesta a los retos planteados por los enfoques tradicionales, promoviendo principios como la adaptabilidad, la iteración y la colaboración constante entre los miembros del equipo.

Dentro de este marco, los \textit{puntos de historia} y el \textit{Planning Poker} representan herramientas clave basadas en teorías de estimación relativa y trabajo en equipo. Estas técnicas permiten abordar problemas complejos mediante la descomposición de tareas y la construcción de consensos. Sin embargo, su uso plantea interrogantes teóricas importantes, como la subjetividad inherente a las estimaciones, la dependencia de la experiencia del equipo y la falta de modelos estandarizados que permitan extrapolar resultados entre proyectos.

Estudiar estas técnicas desde una perspectiva teórica permite evaluar su validez y utilidad en comparación con enfoques tradicionales de estimación. Además, ayuda a construir un marco conceptual que respalde el diseño e implementación de metodologías ágiles en diversos contextos, enriqueciendo el campo de la gestión de proyectos de software.

\section{Justificación Práctica}

En el ámbito práctico, las metodologías ágiles han ganado una amplia aceptación en la industria del software debido a su capacidad para adaptarse a cambios rápidos en los requerimientos y mejorar la satisfacción del cliente. Sin embargo, la estimación de costos y tiempos sigue siendo un desafío significativo, ya que los errores en esta etapa pueden conducir a sobrecostos, retrasos y la insatisfacción de los clientes.

El uso de \textit{puntos de historia} y \textit{Planning Poker} en la estimación ágil ofrece ventajas prácticas, como fomentar la participación del equipo, promover el consenso y facilitar la planificación iterativa. No obstante, estas técnicas presentan problemas prácticos, como la dificultad de estandarización entre equipos, la variabilidad en las estimaciones debido a diferencias de experiencia y la falta de herramientas para medir su efectividad en términos reales.

Este estudio busca abordar estos problemas, proporcionando un análisis crítico de estas técnicas en entornos reales de desarrollo de software. Sus resultados podrán ofrecer a los equipos de desarrollo recomendaciones prácticas para mejorar la precisión de sus estimaciones, optimizar recursos y aumentar la efectividad en la entrega de proyectos ágiles.
