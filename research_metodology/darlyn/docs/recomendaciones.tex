\chapter{Recomendaciones}

\section{Resumen}
En función de los hallazgos obtenidos en el análisis, se proponen las siguientes recomendaciones para optimizar el uso de \textit{puntos de historia} y \textit{Planning Poker} como técnicas de estimación de costos en proyectos de desarrollo de software:

\begin{enumerate}
    \item \textbf{Refinamiento de Historias de Usuario:}
    \begin{itemize}
        \item Implementar sesiones de refinamiento de historias de usuario antes del proceso de estimación, asegurando que los requerimientos sean claros, precisos y desglosados.
        \item Utilizar criterios como \textit{INVEST} (Independiente, Negociable, Valiosa, Estimable, Pequeña y Verificable) para mejorar la calidad de las historias de usuario.
    \end{itemize}

    \item \textbf{Capacitación del Equipo:}
    \begin{itemize}
        \item Realizar capacitaciones periódicas sobre metodologías ágiles y buenas prácticas de estimación para asegurar la correcta aplicación de las técnicas.
        \item Promover ejercicios de estimación en equipos heterogéneos, facilitando la objetividad y el aprendizaje compartido.
    \end{itemize}

    \item \textbf{Uso de Datos Históricos:}
    \begin{itemize}
        \item Incorporar el uso de datos históricos y métricas cuantitativas, como la velocidad del equipo, para validar y ajustar las estimaciones.
        \item Utilizar herramientas de gestión como \textit{Jira} o \textit{Trello} para almacenar y analizar patrones de desempeño.
    \end{itemize}

    \item \textbf{Reducción de la Subjetividad:}
    \begin{itemize}
        \item Fomentar la participación equitativa durante el \textit{Planning Poker} para evitar la influencia de opiniones dominantes.
        \item Complementar el \textit{Planning Poker} con técnicas alternativas, como la estimación por rangos o el uso de puntos de confianza.
    \end{itemize}

    \item \textbf{Monitoreo y Retroalimentación Continua:}
    \begin{itemize}
        \item Implementar un proceso de revisión al final de cada iteración, comparando las estimaciones iniciales con el esfuerzo real.
        \item Fomentar la retroalimentación constante para identificar dificultades y mejorar la aplicación de estas técnicas.
    \end{itemize}

    \item \textbf{Adopción de Herramientas Tecnológicas:}
    \begin{itemize}
        \item Incorporar herramientas tecnológicas como \textit{Jira}, \textit{Monday.com} o \textit{Azure DevOps} para facilitar el seguimiento y análisis de las estimaciones.
    \end{itemize}
\end{enumerate}

Estas recomendaciones están orientadas a reducir las limitaciones observadas, como la subjetividad en las estimaciones y las historias de usuario mal definidas, mejorando así la precisión y efectividad de las técnicas ágiles de estimación en proyectos de desarrollo de software.
