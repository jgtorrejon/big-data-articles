\chapter{Metodología}

\section{Enfoque de la Investigación}

Este estudio adopta un enfoque cuantitativo para analizar la efectividad de las técnicas ágiles, como los puntos de historia y el Planning Poker, en la estimación de costos en proyectos de desarrollo de software. Los datos recopilados se centrarán en métricas como precisión en las estimaciones, tiempos de desarrollo y costos asociados, comparando estas técnicas con métodos tradicionales de estimación.

El análisis incluirá encuestas estructuradas y estudios de caso realizados con equipos de desarrollo de software, permitiendo identificar patrones y correlaciones significativas. Este enfoque garantizará la validez de los resultados al proporcionar una base sólida y objetiva para evaluar las ventajas y limitaciones de los métodos de estimación ágiles.

\section{Diseño de la Investigación}

El diseño metodológico es de tipo no experimental y transeccional descriptivo. Las variables se observarán en su estado natural, sin intervención, permitiendo describir las características y tendencias actuales en el uso de técnicas ágiles de estimación. La recolección de datos se llevará a cabo en un único momento, enfocándose en la precisión y eficiencia de las estimaciones realizadas con puntos de historia y Planning Poker.

Este diseño permitirá detallar patrones que expliquen cómo estas técnicas influyen en la planificación y ejecución de proyectos de desarrollo de software, contribuyendo a decisiones estratégicas más informadas.

\section{Población}

La población objetivo incluye desarrolladores y tech leads con experiencia en metodologías ágiles dentro de equipos de desarrollo de software. Se seleccionarán profesionales que hayan utilizado técnicas como los puntos de historia y el Planning Poker en sus proyectos recientes, asegurando que sus perspectivas sean representativas del sector.

\section{Muestra}

La muestra estará conformada por 30 participantes seleccionados intencionadamente de entre empresas tecnológicas que utilicen metodologías ágiles. Esta selección garantizará diversidad en las perspectivas y experiencias, facilitando un análisis detallado de las prácticas relacionadas con la estimación de costos en entornos ágiles.

\section{Técnicas e Instrumentos}

Las técnicas utilizadas serán encuestas estructuradas y revisión documental de proyectos realizados. El instrumento principal será un cuestionario diseñado específicamente para este estudio, con preguntas cerradas y escalas tipo Likert. Este cuestionario recopilará datos sobre precisión de las estimaciones, tiempos de desarrollo y costos asociados.

Además, se empleará una ficha de análisis documental para obtener información cuantitativa sobre los proyectos evaluados, como el número de integrantes del equipo y el tiempo dedicado a la estimación.

\section{Elaboración del Instrumento}

El cuestionario fue desarrollado considerando las necesidades del estudio y las características de la población objetivo. Las preguntas se diseñaron para abarcar dimensiones como precisión, tiempos y eficiencia en la estimación de costos. Antes de su implementación, el cuestionario fue sometido a una prueba piloto para garantizar su claridad y validez.
