\newpage
\section{Comparación de Productividad entre Dos Equipos de Ventas}

\subsection{Enunciado}
\textbf{Contexto:}
En los datos reales que tienen, identifiquen 2 grupos, por ejemplo podría existir la variable ciudad y
escogen dos ciudades (Equipo A – ciudad A y Equipo B- ciudad B) que trabajan en diferentes regiones. La gerencia quiere comparar
la productividad de los dos equipos para determinar si existe una diferencia significativa en el número de ventas
o generación de algún servicio según sus datos.

\textbf{Instrucciones:}
Obtén los datos de ventas o generación de algún servicio para una muestra para cada equipo.
Realiza una prueba t para dos muestras independientes para comparar la media de ventas de ambos equipos.

\textbf{Hipótesis:}
H0: No hay diferencia en la media de ventas entre los dos equipos.
H1: Existe una diferencia significativa en la media de ventas entre los dos equipos.

\textbf{Indicaciones:}
Usa un nivel de significancia de \( \alpha = 0.05 \)
\\
Compara los resultados de ambas muestras e interpreta el p-valor para tomar una decisión informada.

\subsection{Muestra}
Nosotros tomamos un dataset que se enfoca en el análisis satisfación de empleados \cite{ibm-hr} y
análisis de deserción de empleados, dentro de nuestro grupo de empleados en el dataset tenemos
empleados en industrias de:
\begin{itemize}
    \item Recursos Humanos
    \item Investigación y Desarrollo
    \item Ventas
\end{itemize}
Para el propósito de esta prueba, y como se entiende, lo que se quiere es analizar y comparar
2 equipos de un mismo sector en particular, por lo que se tomará el sector de
\textbf{Investigación y Desarrollo}.

Para esto entonces, vamos a evaluar: \textbf{Satisfacción de empleados de nivel de senior 1 y 5,
para los empleados de Investigación y Desarrollo}.

Como organizamos ambos grupos de investigación:

\begin{lstlisting}[language=Python]
import pandas as pd

df = pd.read_csv("sample_data/HR Employee Attrition.csv")

DEPARTMENT = "Research & Development"
GROUP_SENIOR_1 = 1
GROUP_SENIOR_2 = 5

satisfaction_group_1 = df.loc[(df['Department'] == DEPARTMENT) & (df['JobLevel'] == GROUP_SENIOR_1), 'JobSatisfaction']
satisfaction_group_2 = df.loc[(df['Department'] == DEPARTMENT) & (df['JobLevel'] == GROUP_SENIOR_2),  'JobSatisfaction']
\end{lstlisting}

\subsection{Prueba T}

\begin{lstlisting}[language=Python]
import scipy.stats as stats

t_stat, p_valor = stats.ttest_ind(satisfaction_group_1, satisfaction_group_2)

print(f"Estadistico t: {t_stat:.4f}")
print(f"Valor p: {p_valor:.4f}")

# Decision basada en el p-valor
alpha = 0.05
if p_valor < alpha:
    print("Rechazamos la hipotesis nula: Existe una diferencia significativa en la media de ventas entre los dos equipos.")
else:
    print("No rechazamos la hipotesis nula: No hay diferencia en la media de ventas entre los dos equipos.")    
\end{lstlisting}


\subsection{Resultados}
Una vez realizado el cálculo de la prueba t, obtenemos los siguientes resultados:

\textbf{Estadístico t: -0.0118}
\\
\textbf{Valor p: 0.9906}
\\
\textbf{No rechazamos la hipótesis nula: No hay diferencia en la media de ventas entre los dos equipos.}