\newpage
\section{Comparación de la Satisfacción del Cliente en Tres Sucursales}
\subsection{Enunciado}
\textbf{Contexto:}
Una cadena de tiendas quiere comparar el nivel de satisfacción del cliente entre tres sucursales (Cochabamba, La Paz y Santa Cruz).
La satisfacción del cliente se mide en una escala de 1 a 10 (o el valor de la variable cuantitativa que identifiquen en cada sucursal,
y la empresa desea saber si existen diferencias significativas en la satisfacción entre las sucursales o la propiedad que están estudiando.

\textbf{Instrucciones:}
Obtén una muestra de puntajes de satisfacción del cliente o la variable que están analizando, para cada una de las tres sucursales.
Realiza un ANOVA de una vía para evaluar si hay diferencias significativas en el nivel de satisfacción entre las tres sucursales.

\textbf{Hipótesis:}
H0: Las medias de satisfacción del cliente son iguales en las tres sucursales.
H1: Al menos una de las medias de satisfacción del cliente es diferente entre las sucursales.

\textbf{Indicaciones:}
Usa un nivel de significancia de \(\alpha=0.05.\)
