\newpage
\section{Regresión Lineal}
\subsection{Enunciado}
\textbf{A:} Completar la segunda parte del modelo lineal presentado en clases.
\begin{figure}[h!]
    \centering
    \includegraphics[width=1\textwidth]{images/statement.png}
    \caption{Enunciado de regresión lineal}
    \label{fig:regresion_lineal_statement}
\end{figure}

\begin{verbatim}
municipios = pd.read_csv('mun_cbba/Mun_Cbba2.csv', sep=';', decimal=',')
modelo = smf.ols(formula="tmi ~ anfe + purb", data=municipios).fit()

print(modelo.summary())

residuos = modelo.resid
stats.probplot(residuos, dist="norm", plot=plt)
plt.title("Gráfico Q-Q de los Residuos")
plt.show()

stat, p_value = shapiro(residuos)
print('Estadístico de Shapiro-Wilk:', stat)
print('p-value:', p_value)

print(modelo.summary())
\end{verbatim}

Del resultado el modelo considerando los coeficientes se obtiene que el modelo es:
$$
\text{TMI} = 34.1012 + 1.3292 \cdot \text{anfe} + 0.0505 \cdot \text{purb}
$$



Se realiza una evaluación teórica.

Se planteó que las condiciones de vida y el grado de urbanización implicaban un impacto sobre la mortalidad infantil:

a) Mayor tasa de analfabetismo femenino (anfe) implica una mayor tasa de mortalidad infantil. El signo del coeficiente debe ser positivo. En el modelo estimado, el coeficiente de anfe anfe es  1.3292 1.3292, lo cual es consistente con la teoría. Este coeficiente es estadísticamente significativo ( 𝑝 < 0.001 p<0.001), lo que indica que existe evidencia empírica para concluir que el analfabetismo femenino tiene un impacto positivo y significativo sobre la mortalidad infantil.

b) Mayor porcentaje de población urbana (purb) implica una mayor tasa de mortalidad infantil. El signo del coeficiente también debe ser positivo. En el modelo estimado, el coeficiente de purb purb es  0.0505 0.0505, pero este valor no es estadísticamente significativo ( 𝑝 = 0.588 > 0.05 p=0.588>0.05). Esto sugiere que no hay evidencia empírica suficiente para concluir que el grado de urbanización afecta significativamente la mortalidad infantil.

El modelo estimado constituye una evidencia empírica parcial de la teoría planteada en el primer paso: mientras que el analfabetismo femenino demuestra ser un factor significativo, el grado de urbanización no tiene un impacto concluyente en este caso.



Evaluacion de la bondad de ajuste

En este ejemplo, el indicador es 𝑅 2 = 0.726 R2=0.726, lo que significa que el 72.6% de la variabilidad de la mortalidad infantil está explicada por la asociación lineal con el analfabetismo femenino ( anfe anfe) y el grado de urbanización ( purb). El restante 27.4% se debe a otros factores no considerados en este modelo.

\textbf{B:} Generar una regresión lineal múltiple con 3 variables: dos  independiente y 1 dependientes (variable target, objetivo). Puede usar las bases de datos reales que uso en las prácticas anteriores.