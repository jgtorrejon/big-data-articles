\chapter{Reflexión Personal}

A la edad que tengo, ya con 31 años, y al ver estas herramientas, me he dado cuenta de que he perdido mucho tiempo y que pude haber aprendido mucho más de lo que hice. No quiere decir que ya no pueda hacerlo, sino que con el paso de los años se vuelve más difícil volver a empezar. El fundador de Rappi comenzó su camino a los 20 años, y Freddy Vega, creador de Platzi, también inició temprano su recorrido por el mundo del emprendimiento y la tecnología. Los cito a ellos, y no a otros nombres más grandes, por la cercanía geográfica y cultural que existe, porque representan que no es necesario nacer en Silicon Valley para cambiar el mundo, sino tener decisión y propósito desde temprano.

Quizás el motivo de no haber comenzado antes, en mi caso, tiene que ver con mi entorno social, educativo y cultural, donde no se fomentaba tanto la innovación ni el pensamiento emprendedor. Tal vez también, como muchos, busqué estabilidad antes que propósito, comodidad antes que crecimiento. O quizás —como dicen algunos— esas son las excusas que uno se inventa cuando no logra sus objetivos a tiempo. No lo sé, y puede que nunca lo sepa. Sin embargo, lo que sí sé con certeza es que al conocer y analizar estas herramientas, he comprendido que aún me falta mucho por recorrer. Ha sido un verdadero choque profesional y personal, una sacudida que me ha hecho replantear mis prioridades y la manera en la que enfrento la vida. Pensaba que estaba en una etapa cómoda, que a mi edad ya había alcanzado varios propósitos y logros personales, pero hoy entiendo que el verdadero crecimiento no se mide por lo que uno acumula, sino por la capacidad de seguir evolucionando.

Estas herramientas me demostraron que debo continuar iterando sobre mí mismo hasta el final, que la versión actual de quien soy no debe ser estática, sino un punto de partida. He comprendido que no puedo considerarme una persona emprendedora si primero no defino mi interior, si no entiendo mis valores, mis motivaciones y mi verdadero propósito. Ser emprendedor no se trata solo de crear empresas o innovar en tecnología, sino de transformar la forma en la que uno piensa, siente y actúa.

Me costó mucho realizar mi Plan de Vida, ese ejercicio de sintetizar mi propósito personal en una sola página. Fue un proceso de introspección profundo, porque me obligó a mirar hacia adentro y reconocer mis debilidades, mis miedos y también mis fortalezas. Descubrí que muchas veces he actuado por inercia, no por convicción; que me he dejado llevar por las circunstancias en lugar de construirlas. Pero también descubrí que dentro de mí existe un deseo enorme de superación, una voz que me dice que todavía estoy a tiempo de construir algo significativo.

Hoy no veo mis treinta y un años como una barrera, sino como el punto exacto en el que mi experiencia y mi conciencia se encuentran para comenzar una nueva etapa. Mi reflexión final es que nunca es tarde para reinventarse, pero sí es urgente empezar. Porque el tiempo no espera, y la única forma de honrarlo es usándolo para crecer y dejar huella.