\chapter{Preguntas}
En este capítulo estaremos respondiendo todas las preguntas de la mentoría.
\section{¿Cuál fue el quiebre declarado por el coachee?}
El quiebre fue interno y profesional.
Ella siente que, aunque ha hecho las cosas bien en su vida, no está segura de si realmente es capaz de liderar. Hay una duda constante sobre su propio valor en roles de liderazgo.

\section{¿Cuáles eran las interpretaciones que la llevaron a este quiebre?}

Las creencias que ella misma sostenía eran más o menos así:

“Yo controlo todo porque si no lo hago, las cosas no salen bien.”

“No sé si realmente estoy preparada para liderar.”

“Tengo varias versiones de mí, no sé cuál es la real para usarla en un rol de liderazgo.”

“Nunca me fue mal… pero tampoco sé si soy suficiente para avanzar.”

Estas creencias la mantienen insegura y le generan esa duda constante.

\section{¿Qué nuevas interpretaciones surgieron durante la conversación?}

Durante la sesión, empezó a ver que:

No tiene que pelear contra sus distintas facetas, sino observarlas.

Su capacidad de adaptarse no es una debilidad, puede ser una fortaleza.

El liderazgo no es algo que “se demuestra de golpe”, sino que se construye con pasos concretos.

Lo importante no es controlar todo, sino darse cuenta de cómo actúa y elegir.

\section{Preguntas más relevantes que hice como coach}

¿Qué es lo que te hace dudar de tu capacidad para liderar?

Cuando dices que “usas diferentes personalidades”, ¿qué estás queriendo proteger?

¿Cómo te gustaría sentirte cuando estés en un rol de liderazgo?

¿Qué momentos del día podrías usar para observarte sin juzgarte?

¿Qué sería un primer paso pequeño, concreto y medible hacia tus metas?

\section{¿Qué acciones se hicieron posibles ahora?}

Después de la conversación, ella vio posible:

Observarse a sí misma en situaciones reales (con alarmas).

Plantear metas claras en vez de solo decir “quiero cambiar”.

Tomar decisiones pequeñas pero firmes hacia su desarrollo (título, área profesional y salud).

\section{Estado de ánimo al inicio y al final}

Al inicio: Estaba insegura, hablaba mucho para explicarse, como queriendo justificar lo que sentía. Su cuerpo estaba algo tenso.

Al final: Se mostró más tranquila y con claridad, con un tono más firme. Ya no hablaba desde la duda, sino desde la intención de actuar.

\section{¿Se comprometió a alguna acción?}

Sí.
Se comprometió a:

Ponerse alarmas durante el día para observar cómo actúa en situaciones distintas.

Definir sus objetivos en formato SMART, especialmente:

Sacar su título

Postular a Recursos Humanos, aunque sea como Junior

Comenzar con la dieta
Se espera que comience esto inmediatamente, en el transcurso de esta semana.

\section{Reflexión personal como coach}

Lo que aprendí en esta sesión es que escuchar de verdad es más importante que apurarme en decir algo. Cuando la dejé hablar y simplemente estuve presente, ella misma fue llegando a su claridad. Entendí que no tengo que “resolverle” el problema, sino acompañarla a que lo vea desde otro ángulo. Me ayudó también a bajar mis propios juicios y estar más atento a su emoción y su cuerpo, no solo a sus palabras.