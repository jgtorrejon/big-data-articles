\section{Reflexión Personal}
Este libro me hizo pensar bastante en mi forma de estar en el trabajo y en la vida. Me quedó claro que el desarrollo de habilidades gerenciales no es aprender a mandar, ni a organizar, ni a hablar bonito. Es aprender a conocerse y darse cuenta de qué cosas uno hace por miedo, por costumbre o simplemente porque así lo vio siempre.

Lo que más me pegó es ver cómo muchas personas, al igual que el protagonista, viven automáticas, tratando de encajar, usando “corbatas” para parecer profesionales, responsables o exitosos, mientras por dentro sienten otra cosa. Me di cuenta de que muchas veces yo también hago eso: soy serio, eficiente y responsable, pero a veces actúo en piloto automático, sin detenerme a preguntarme si esto realmente tiene sentido para mí.

El libro me recordó que liderar empieza por uno mismo. No puedo querer acompañar o guiar a otros si yo mismo me llevo a los empujones, si me hablo mal internamente, si no confío en mis decisiones o si quiero controlar todo para que nada salga mal. Entendí que el control no es liderazgo. El control viene del miedo. Y el liderazgo viene de la confianza.

Algo que también me hizo ruido fue el tema de la escucha. Yo escucho, pero muchas veces escucho solo para responder o para arreglar el problema rápido. Y no siempre es eso lo que la otra persona necesita. A veces solo necesita sentirse tomada en cuenta, que alguien realmente esté presente. Esto lo puedo aplicar en mi trabajo y en mi casa. Me he dado cuenta de que cuando escucho con calma, la conversación cambia, baja la tensión y la otra persona se siente acompañada.

También entendí que el ambiente de trabajo no se cambia por discursos ni políticas. Se cambia por pequeñas acciones constantes: agradecer, reconocer el esfuerzo, ser claro con lo que uno espera, pedir ayuda cuando corresponde, no humillar o hablar desde la soberbia. Todos esos detalles construyen o destruyen un ambiente laboral.

En mi día a día, quiero aplicar lo siguiente:

Tomarme unos segundos antes de reaccionar, especialmente si estoy cansado o estresado.

Delegar más sin sentir que si no lo hago yo, no va a salir bien.

Ser más genuino cuando hablo con alguien, sin máscaras innecesarias.

Recordar que lo importante no es solo lo que hago, sino cómo lo hago y cómo dejo sentir al otro.

Al final, el libro me hizo darme cuenta de algo simple: un buen líder es un ser humano que se permite ser humano, y eso empieza por mirarse hacia adentro con honestidad. Y eso sí me hizo sentido. Es por ahí.