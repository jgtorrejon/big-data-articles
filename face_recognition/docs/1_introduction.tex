\section{Introducción}

% TODO El abstract seria autogenerarlo una vez se tenga la conclusion pero tengo mi duda si en latex es una seccion mas o un apartado especial.

El reconocimiento automático de rostros ha sido un campo ampliamente explorado en visión por computadora; sin embargo, su aplicación en contextos específicos sigue siendo un desafío técnico y socialmente relevante. En este proyecto nos enfocamos en el \textbf{análisis de atributos faciales}, en particular la \textbf{clasificación de género} y la \textbf{predicción de edad} a partir de imágenes de rostros. Si bien el término ``detección de rostros'' abarca múltiples líneas de investigación, hemos acotado el problema hacia un enfoque de clasificación y regresión dentro de un mismo modelo, buscando así una solución más concreta y alcanzable en el marco de este módulo.

En el \textbf{contexto boliviano}, este tipo de sistemas tiene diversas aplicaciones potenciales. Por ejemplo, podrían emplearse en entornos de \textbf{seguridad ciudadana}, apoyando a sistemas de control de accesos o cámaras de videovigilancia mediante la extracción de metadatos demográficos básicos. Asimismo, pueden ser útiles en áreas de \textbf{salud y educación}, donde la estimación automática de la edad podría facilitar la detección temprana de irregularidades en registros poblacionales o en la provisión de servicios diferenciados por rango etario. Estos casos de uso, aunque aún preliminares, demuestran que contar con un modelo robusto de clasificación y predicción facial puede constituirse en un componente fundamental para futuros proyectos de visión por computadora.

La elección de \textbf{deep learning} como enfoque metodológico se justifica en virtud de la complejidad inherente al problema. Las características faciales presentan gran variabilidad debido a factores como iluminación, ángulo de captura, expresiones faciales y diversidad étnica. Los modelos tradicionales de aprendizaje automático suelen tener dificultades para manejar estas variaciones; en contraste, las \textbf{redes neuronales profundas} permiten aprender representaciones jerárquicas de alto nivel directamente desde los datos, logrando así un desempeño significativamente superior en tareas de clasificación y regresión basadas en imágenes.

De manera preliminar, los experimentos realizados muestran que es factible entrenar un único modelo con dos salidas: una para la clasificación de género y otra para la predicción de edad. Este enfoque \textbf{multitarea} no solo optimiza el uso de los recursos computacionales, sino que también abre la posibilidad de extender el trabajo hacia otros atributos faciales en futuros proyectos académicos, particularmente en la asignatura de visión por computadora que se encuentra pendiente en el plan de estudios de nuestra maestría.
